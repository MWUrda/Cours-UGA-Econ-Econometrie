\input{../../Preambule}

\usepackage{tikz}
\usepackage{enumitem}


\usepackage{fancyhdr}
\pagestyle{fancy}
%\renewcommand{\subsection{mark}[1]{\markright{#1}{}}
\fancyhead{}
\fancyfoot{} 
%\fancyhead[LE,LO]{\tiny{\thepage}}
\fancyhead[C]{\small\textsc{Économétrie}}
%fancyhead[CE,CO]{\tiny{\rightmark}}
\fancyhead[L]{\small\textsc{UGA}}
\fancyfoot[C]{\small{\thepage}}
%\fancyfoot[R]{\small \textcopyright \ \  \small\textsc
\fancyhead[R]{ \small\textsc{M. Urdanivia}}
%\renewcommand{\headrulewidth}{0pt}
\renewcommand{\footrulewidth}{0pt}

%\pagenumbering{roman}


\begin{document} 
\usetikzlibrary{positioning}
\usetikzlibrary{snakes}
\usetikzlibrary{calc}
\usetikzlibrary{arrows}
\usetikzlibrary{decorations.markings}
\usetikzlibrary{shapes.misc}
\usetikzlibrary{shapes}
%\tikzset{block/.style={draw, rectangle, fill=gray!20, 
%\tikzset{empty/.style={draw, rectangle, fill=none, tex
%\tikzset{line/.style={draw, -latex'}}
%\onehalfspace

\begin{titlepage}
\centering
	%\includegraphics[width=0.15\textwidth]{logoUGA2020
	{\scshape\Large \textsc{Économétrie: UGA}\par}
	\vspace{1cm}
	%{\scshape\large \textsc{Extremum Estimators(1)}\par}
	%\vspace{1cm}
	{\Large\bfseries \textsc{4: SYSTÈMES D’ÉQUATIONS SIMULTANÉES} \par}
    \vspace{1cm}   
    {\Large\bfseries \textsc{Application} \par}
	\vspace{1cm}
	{(\textsc{Cette version: \today})\par}
	\vspace{1cm}
	{\large \textsc{Michal Urdanivia}
	\footnote{Contact:  
	\href{mailto:michal.wong-urdanivia@univ-grenoble-alpes.fr}{michal.wong-urdanivia@univ-grenoble-alpes.fr}, 
	 Université de Grenoble Alpes,  Faculté d'\'Economie, GAEL.}\par}
	%\vfill
	%supervised by\par
	%Dr.~Mark \textsc{Brown}
%\vfill
% Bottom of the page
	
\end{titlepage}


\newpage

\tableofcontents

\newpage

\section{Équations simultanées et 2MC }

Les questions suivantes sont issues de \cite{Wooldridge2010}(Problème 9.13, chapitre 9). 
Il s'agit d'une application sur des données et concernant une problématique développé par \cite{Romer1993QJE}. Les données peuvent être obtenues ici: 

\medskip

\url{http://qcpages.qc.cuny.edu/~rvesselinov/statadata/OPENNESS.DTA}

\medskip

L'objectif est d'étudier est d'étudier l'effet causal de l'ouverture d'une économie sur l'inflation. 
Le modèle considéré est le modèle à équations simultanées suivant:

\begin{align*}
	y_{(in,i)} &= a_0 + b_0 y_{(op, i)} + d_0 q_i + u_{(in,i)}\\
	y_{(op,i)} &= \alpha_0 + \beta_0 y_{(inf, i)} + \delta_0 q_i  + \gamma_0q_{(op, i)}+ u_{(op,i)}
\end{align*}
où pour un pays $i$: $y_{(in,i)}$ est le taux d'inflation("$inf$" dans le données), 
$y_{(op,i)}$ est une mesure de son degré d'ouverture("$open$" dans les données) , 
$q_i$ est son revenu par habitant("$pinc$" dans les données), et $q_{(op, i)}$ 
est la superficie en log("$lland$" dans les données). Le revenu par habitant, et la superficie sont supposées exogènes.

\begin{enumerate}
\item Expliquez pourquoi une estimation par MCO de $b_0$ dans la première équation ne permet pas de mesurer 
l'effet de du degré d'ouverture sur l'inflation?
\item À quelle condition la première équation est identifiée?
\item Estimez la forme réduite du degré d'ouverture sur les variables exogènes, et commentez l'effet de la superficie.
\item Sur la base du résultat précédent proposez une stratégie d'estimation de la première équation par 2MC.
 Comparez le résultat de cette estimation à ceux obtenus par MC0.  
 \item On ajoute à la première équation le terme $c_0 y_{(op, i)}^2$. Expliquez 
 pourquoi on a besoin d'une VI supplémentaire. Proposez en une.
 \item Estimez le modèle indiqué à la question précédente.
\end{enumerate}

\section{Équations simultanées et MMG}

On considère un modèle d'offre et demande s'inspirant du travail de \cite{AGI_RES_2000}. 
On utilise les mêmes données qui peuvent être obtenues ici:

\medskip

\url{https://people.brandeis.edu/~kgraddy/datasets/fishdata.dta}

\medskip

Le but est d'étudier la demande de poisson merlan sur le "Fulton Fish Market". 
Le modèle étudiée est le suivant:

\begin{align*}
	y_{d, i} & = a_0 p_i + \boldb_0^\prime z_{d, i} +  \boldd_0^\prime w_i + u_{d, i}\\
	y_{o, i} & = \alpha_0 p_i + \boldbeta_0^\prime z_{o, i} + \boldgamma_0^\prime w_i + u_{o, i},
\end{align*}

où la première équation est une équation de demande et la deuxième une équation d'offre avec $y_{d, i}$ la quantité demandé
et $y_{o, i}$ la quantité offerte, $p_i$ le prix $z_{d, i}$ et $z_{o, i}$, des 
vecteurs de régresseurs propres à chaque équation et $w_i$ des régresseurs communs aux deux équations.
On observe les quantités à l'équilibre telles que $y_{d, i} = y_{o, i} = y_i$ de sorte que le modèle
s'écrit:

\begin{align*}
	y_i & = a_0 p_i + \boldb_0^\prime z_{d, i} +  \boldd_0^\prime w_i + u_{d, i}\\
	y_i & = \alpha_0 p_i + \boldbeta_0^\prime z_{o, i} + \boldgamma_0^\prime w_i + u_{o, i},
\end{align*}











\bibliographystyle{jpe}
\bibliography{../../Biblio.bib}
 \end{document}
