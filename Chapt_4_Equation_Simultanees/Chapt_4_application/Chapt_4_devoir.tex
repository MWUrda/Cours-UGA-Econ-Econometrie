\input{../../Preambule}

\usepackage{tikz}
\usepackage{enumitem}


\usepackage{fancyhdr}
\pagestyle{fancy}
%\renewcommand{\subsection{mark}[1]{\markright{#1}{}}
\fancyhead{}
\fancyfoot{} 
%\fancyhead[LE,LO]{\tiny{\thepage}}
\fancyhead[C]{\small\textsc{Économétrie}}
%fancyhead[CE,CO]{\tiny{\rightmark}}
\fancyhead[L]{\small\textsc{UGA}}
\fancyfoot[C]{\small{\thepage}}
%\fancyfoot[R]{\small \textcopyright \ \  \small\textsc
\fancyhead[R]{ \small\textsc{M. Urdanivia}}
%\renewcommand{\headrulewidth}{0pt}
\renewcommand{\footrulewidth}{0pt}

%\pagenumbering{roman}


\begin{document} 
\usetikzlibrary{positioning}
\usetikzlibrary{snakes}
\usetikzlibrary{calc}
\usetikzlibrary{arrows}
\usetikzlibrary{decorations.markings}
\usetikzlibrary{shapes.misc}
\usetikzlibrary{shapes}
%\tikzset{block/.style={draw, rectangle, fill=gray!20, 
%\tikzset{empty/.style={draw, rectangle, fill=none, tex
%\tikzset{line/.style={draw, -latex'}}
%\onehalfspace

\begin{titlepage}
\centering
	%\includegraphics[width=0.15\textwidth]{logoUGA2020
	{\scshape\Large \textsc{Économétrie: UGA}\par}
	\vspace{1cm}
	%{\scshape\large \textsc{Extremum Estimators(1)}\par}
	%\vspace{1cm}
	{\Large\bfseries \textsc{4: SYSTÈMES D’ÉQUATIONS SIMULTANÉES} \par}
    \vspace{1cm}   
    {\Large\bfseries \textsc{Devoir 1} \par}
	\vspace{1cm}
	{(\textsc{Cette version: \today})\par}
	\vspace{1cm}
	{\large \textsc{Michal Urdanivia}
	\footnote{Contact:  
	\href{mailto:michal.wong-urdanivia@univ-grenoble-alpes.fr}{michal.wong-urdanivia@univ-grenoble-alpes.fr}, 
	 Université de Grenoble Alpes,  Faculté d'\'Economie, GAEL.}\par}
	%\vfill
	%supervised by\par
	%Dr.~Mark \textsc{Brown}
%\vfill
% Bottom of the page
	
\end{titlepage}


\newpage

\tableofcontents

\newpage

\section{Objectifs}
\begin{enumerate}
\item Illustrer les mécanismes d’identification dans les systèmes d’équations simultanées et leurs liens avec les techniques de VI
\item Manipuler des notations matriciels
\item Montrer comment utiliser un résultat théorique en pratique, ici la méthode du delta pour déterminer la distribution as. 
d’un estimateur des MCI.
\item Appliquer sur des données certaines des méthodes.
\end{enumerate}

\section{Théorie}

\subsection{Modèles récursifs}
Les vecteurs de variables aléatoires $(y_{1i},y_{2i},z_{1i},z_{2i})$ sont indépendants, identiquement distribués pour 
$i= 1,\ldots, N$. Ils satisfont en outre les conditions de régularité requises par les applications de la loi
 des grands nombres et du théorème central limite nécessaires pour la normalité as. des estimateurs considérés 
 (lorsqu’ils sont employés de manière pertinente!). 

 On considère ici différentes version du modèle suivant:

    
 \begin{align}
    \left\{
 \begin{array}{ll}
    y_{1i} &= a_{10}z_{1i} + a_{2,0}z_{2i} + a_{3,0}y_{2i} +u_{1i}\\
    y_{2i}&= b_{1,0}z_{1i} + b_{2,0}z_{2i}+u_{2i}
 \end{array}
\right.
&, \ \text{avec} \  \Exp[\boldu_i|\boldz_i] =0,  \text{et} \ \Exp[\boldu_i\boldu_i^\prime|\boldz_i]=\boldOmega
\label{eq1}
 \end{align}

 avec les notations suivantes:

 \begin{align*}
    \boldy_i \equiv 
    \left[
    \begin{array}{l}
        y_{1i}\\
        y_{2i}
    \end{array}
    \right]
    , \quad 
    \boldz_i \equiv 
\left[
\begin{array}{l}
    z_{1i}\\
    z_{2i}
\end{array}
\right]
, \quad 
\boldu_i \equiv 
\left[
\begin{array}{l}
    u_{1i}\\
    u_{2i}
\end{array}
\right]
, \quad 
\boldOmega \equiv
\left[
\begin{array}{ll}
    \omega_{11}&\omega_{12}\\
    \omega_{21}&\omega_{22}
\end{array}
\right]
 \end{align*}
    
 On utilisera également les notations:


 \begin{align*}
    \bolda_0 \equiv 
    \left[
    \begin{array}{l}
        a_{1,0}\\
        a_{2, 0}\\
        a_{3, 0}
    \end{array}
    \right]
    , \quad 
    \boldb_0 \equiv 
 \left[
 \begin{array}{l}
    b_{1,0}\\
    b_{2,0}
\end{array}
 \right]
\end{align*}

et:

\begin{align*}
    \underline{\boldy}_k \equiv 
    \left[
    \begin{array}{c}
        y_{k,1}\\
        \vdots\\
        y_{k, N}
    \end{array}
    \right]_{N\times 1}
    , \quad 
    \underline{\boldz}_k \equiv
    \left[
    \begin{array}{c}
        z_{k,1}\\
        \vdots\\
        z_{k, N}
    \end{array}
    \right]_{N\times 1}
    , \quad \text{pour $k=1, 2$ et} \
    \boldZ \equiv 
    \left[
\begin{array}{ll}
    \underline{\boldz}_1& \underline{\boldz}_2
\end{array}
\right]_{N\times 2},
\end{align*}
et on suppose que $\Var[\boldz_i]$ est inversible.

\begin{enumerate}
    \item Justifier l’absence de paramètres constants pour les modèles de $y_{1i}$ et $y_{2i}$ 
    \item  Déterminer les variables exogènes et endogènes de chacune des équations
    du système et du système
    \item Qualifier le modèle décrit par le système d’équations \eqref{eq1}.
    \item Analyser l’identification de $\boldb_0$ et $\bolda_0$ dans le modèle.  
    \item Proposer un estimateur convergent de $\boldb_0$ et donner sa forme avec les $y_{2,i}$ et $\boldz_i$ 
    ainsi qu’avec $\underline{\boldy}_k$ et $\boldZ$.
    \item Donner une approximation de la distribution de $\boldb_0$ lorsque $N$ est grand et un estimateur de sa précision.
\end{enumerate}

\bibliographystyle{jpe}
\bibliography{../../Biblio.bib}
 \end{document}
