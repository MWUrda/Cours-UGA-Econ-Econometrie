\documentclass[10pt, reqno]{amsart}
\input{../Preambule}
\newcommand{\bxi}{ \boldx_i }
\usepackage{tikz}
\usepackage{enumitem}
\usepackage{txfonts}
\usepackage{mathrsfs}
\usepackage{fancyhdr}



\usepackage{fancyhdr}
\usetikzlibrary{calc,decorations.pathmorphing,patterns}
\usepackage{listofitems}
\tikzstyle{mynode}=[thick,draw=blue,fill=blue!20,circle,minimum size=22]
\usepackage{amsaddr}
\usepackage{mathrsfs}

\makeatletter

\pgfdeclaredecoration{penciline}{initial}{
    \state{initial}[width=+\pgfdecoratedinputsegmentremainingdistance,auto corner on length=1mm,]{
        \pgfpathcurveto%
        {% From
            \pgfqpoint{\pgfdecoratedinputsegmentremainingdistance}
                            {\pgfdecorationsegmentamplitude}
        }
        {%  Control 1
        \pgfmathrand
        \pgfpointadd{\pgfqpoint{\pgfdecoratedinputsegmentremainingdistance}{0pt}}
                        {\pgfqpoint{-\pgfdecorationsegmentaspect\pgfdecoratedinputsegmentremainingdistance}%
                                        {\pgfmathresult\pgfdecorationsegmentamplitude}
                        }
        }
        {%TO 
        \pgfpointadd{\pgfpointdecoratedinputsegmentlast}{\pgfpoint{1pt}{1pt}}
        }
    }
    \state{final}{}
}
\makeatother

\pagestyle{fancy}
%\renewcommand{\subsection{mark}[1]{\markright{#1}{}}
\fancyhead{}
\fancyfoot{} 
%\fancyhead[LE,LO]{\tiny{\thepage}}
\fancyhead[C]{\small\textsc{Inférence Statistique}}
%fancyhead[CE,CO]{\tiny{\rightmark}}
\fancyhead[L]{\small\textsc{Université de Grenoble Alpes}}
\fancyfoot[C]{\small{\thepage}}
%\fancyfoot[R]{\small \textcopyright \ \  \small\textsc
\fancyhead[R]{ \small\textsc{M. W. Urdanivia}}
%\renewcommand{\headrulewidth}{0pt}
\renewcommand{\footrulewidth}{0pt}

%\pagenumbering{roman}


\begin{document} 
\usetikzlibrary{positioning}
\usetikzlibrary{snakes}
\usetikzlibrary{calc}
\usetikzlibrary{arrows}
\usetikzlibrary{decorations.markings}
\usetikzlibrary{shapes.misc}
\usetikzlibrary{shapes}
%\tikzset{block/.style={draw, rectangle, fill=gray!20, 
%\tikzset{empty/.style={draw, rectangle, fill=none, tex
%\tikzset{line/.style={draw, -latex'}}
%\onehalfspace

%\includepdf{trame}

\begin{titlepage}
\centering
	%\includegraphics[width=0.15\textwidth]{logoUGA2020}
	{\scshape\Large \textsc{Université de Grenoble Alpes\\(M1 BDA-EETT, S1)}\par}
	\vspace{0.5cm}
	{\Large\bfseries \scshape\Large \textsc{INFÉRENCE STATISTIQUE}\par}
	%{\scshape\large \textsc{Extremum Estimators(1)}\par}
	%\vspace{1cm}
	\vspace{0.5cm}
	{\Large\bfseries \textsc{NOTES SUR LA THÉORIE DE LA DÉCISION ET APPLICATION À LA THÉORIE DES TESTS} \par}
    \vspace{0.5cm}   
    %{\Large\bfseries \textsc{EXAMEN(L3 MIASH, S2)} \par}
	%\vspace{1cm}
	{(\textsc{Cette version: \today})\par}
	\vspace{1cm}
	{\large \textsc{Michal W. Urdanivia}
	\footnote{Contact:  
	\href{mailto:michal.wong-urdanivia@univ-grenoble-alpes.fr}{michal.wong-urdanivia@univ-grenoble-alpes.fr}, 
	 Université de Grenoble Alpes,  Faculté d'\'Economie, GAEL.}\par}
	 %\includegraphics[width=0.15\textwidth]{logoUGA}
	%\vfill
	%supervised by\par
	%Dr.~Mark \textsc{Brown}
%\vfill
% Bottom of the page
	
\end{titlepage}


\newpage

\tableofcontents

\newpage

\section{Introduction} 
On considère un modèle statistique de la forme lm ETC,


\begin{tikzpicture}[decoration=penciline]
	\draw[decorate,style=help lines] (-2,-2) grid[step=1cm] (4,4);
	\draw[decorate,thick] (0,0) -- (0,3) -- (3,3);
	\draw[decorate,ultra thick,blue] (3,3)  arc (0:-90:2cm); %% This is supposed to be an arc!!
	\draw[decorate,thick,pattern=north east lines] (-0.4cm,-0.8cm) rectangle (1.2,-2);
	\node[decorate,draw,inner sep=0.5cm,fill=yellow,circle] (a) at (2,0) {}; %% That's not even an ellipse !!
	\node[decorate,draw,inner sep=0.3cm,fill=red] (b) at (2,-2) {};
	\draw[decorate] (b) to[in=-45,out=45] (a); %% This was supposed to be an edge!!
	\node[decorate,draw,minimum height=2cm,minimum width=1cm] (c) at (-1.5,0) {};
	\draw[decorate,->,dashed] (-0.5cm,-0.5cm) -- (-0.5cm,3.5cm)  -| (c.north);
	\end{tikzpicture}
	
	\newpage
	
	\begin{tikzpicture}[x=2.2cm,y=1.4cm]
	  \readlist\Nnod{4,5,5,5,3} % number of nodes per layer
	  % \Nnodlen = length of \Nnod (i.e. total number of layers)
	  % \Nnod[1] = element (number of nodes) at index 1
	  \foreachitem \N \in \Nnod{ % loop over layers
		% \N     = current element in this iteration (i.e. number of nodes for this layer)
		% \Ncnt  = index of current layer in this iteration
		\foreach \i [evaluate={\x=\Ncnt; \y=\N/2-\i+0.5; \prev=int(\Ncnt-1);}] in {1,...,\N}{ % loop over nodes
		  \node[mynode] (N\Ncnt-\i) at (\x,\y) {};
		  \ifnum\Ncnt>1 % connect to previous layer
			\foreach \j in {1,...,\Nnod[\prev]}{ % loop over nodes in previous layer
			  \draw[thick] (N\prev-\j) -- (N\Ncnt-\i); % connect arrows directly
			}
		  \fi % else: nothing to connect first layer
		}
	  }
	\end{tikzpicture}
	
	
	\begin{align*}
	  x+y&\coloneqq y+z+v+k+f+\boldL\boldM++\boldZ+K+\varepsilon+\gamma+\boldw+\boldv+\boldy+\boldX\cdot\boldz\cdot\boldz\cdot\boldz,\\
	  &\coloneqq g+\varepsilon+ \eta+\mu + \gamma+\eta + \theta + \xi + \phi,\\
	  &=\boldX^\top\boldY\cdot \boldZ\cdot + \boldX+\boldY +  \boldL\boldM +  \boldL\boldM +  \boldL\boldM +  \boldL\boldM + \beta\\
	  &= \boldL\boldM\boldL\boldM,\\
	  &= \boldL\boldM \boldL\boldM,\\
	  &=\boldalpha+\boldbeta+\boldmu+\boldOmega+\beta+\alpha+\gamma+\varepsilon+\gamma+\varepsilon\\
	  & = \varepsilon+\varkappa+\varepsilon+\varepsilon,\\
	  & = \varepsilon+\varkappa+\varepsilon+\varepsilon,\\
	  & = \varepsilon+\varkappa + \varepsilon+\varkappa,\\
	  & = \varepsilon+\varkappa + \varepsilon+\varkappa,\\
	  & = \varepsilon+\varkappa+\varkappa + \boldxi + \boldxi  + \boldxi + \boldxi,\\
	  &\equiv \boldL\boldM +\boldL\boldM,\\
	  &= \boldL\boldM +\boldL\boldM,\\
	  &= \boldL\boldM +\boldL\boldM,\\
      &= \boldL\boldM +\boldL\boldM.\\
	\end{align*}

%\section{Application: salaires et offre de travail}

%On considère le système suivant:

%\begin{align*}
%y_{1i} &= a_{0, 0} + a_{1, 0} y_{2i} +  a_{2, 0}z_{1, i} + a_{3, 0}z_{2, i}
%\end{align*}
%\bibliographystyle{jpe}
%\bibliography{../../Biblio.bib}
 \end{document}
