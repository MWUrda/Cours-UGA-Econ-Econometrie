\documentclass[10pt, reqno]{amsart}
%\documentclass[10pt, reqno]{amsart}
%\numberwithin{equation}{section}
\usepackage[svgnames]{xcolor}
\usepackage{setspace} 
\usepackage{etex}
\usepackage{tikz}
\usepackage{mathrsfs}
\usepackage{todonotes}                         %I use this to include comments on the PDF document. 
                                                                     %Requires todonotes.sty
\usepackage[labelfont=bf, margin=1cm]{caption}

\usepackage{pst-func}
\usepackage[normalem]{ulem}
\usepackage{amsmath}
\usepackage{amsthm}
\usepackage{thmtools}
\usepackage{amsfonts}
\usepackage{amssymb}
\usepackage{graphicx}
\usepackage{hyperref}
%\usepackage{tikz}
\usepackage{mdframed}
\usepackage{showexpl}
%\usepackage{geometry}
%\geometry{verbose,tmargin=1.5cm,bmargin=1.5cm,lmargin=1.5cm,rmargin=1.5cm}
%\usepackage[left=1in,right=1in,top=0.9in,bottom=0.9in]{geometry}
%\usepackage[vmargin={1.5in, 1.5in},hmargin={1.5in, 1.5in}]{geometry} 
\usepackage[vmargin={1.5in, 1.5in},hmargin={1.5in, 1.5in}]{geometry} 
%\usepackage[vmargin={1.5in, 1.5in},hmargin={1.2in, 1.2in}]{geometry} 
%\usepackage{geometry}
%\geometry{verbose,tmargin=3cm,bmargin=3cm,lmargin=3cm,rmargin=3cm}
\usepackage{setspace}
%\usepackage[latin9]{luainputenc}
\setlength{\parskip}{\medskipamount}
%\setlength{\parindent}{0pt}
%\documentclass[12pt, reqno]{amsart}
%\usepackage[svgnames]{xcolor}
%\usepackage{setspace} 
%\usepackage{etex}
%\usepackage{tikz}
%\usepackage{todonotes}                         %I use this to include comments on the PDF document. 
                                                                     %Requires todonotes.sty
%\usepackage[labelfont=bf, margin=1cm]{caption}

%\usepackage{pst-func}
%\usepackage[normalem]{ulem}
%\usepackage{amsmath}
%\usepackage{amsthm}
%\usepackage{amsfonts}
%\usepackage{amssymb}
%\usepackage{graphicx}
%\usepackage{hyperref}
%\usepackage[left=1in,right=1in,top=0.9in,bottom=0.9in]{geometry}
%\usepackage[vmargin={1.5in, 1.5in},hmargin={1in, 1in}]{geometry} 
\usepackage{multirow}
\usepackage{verbatim}
\usepackage[greek,frenchb]{babel}
\usepackage[babel=true,kerning=true]{microtype}
\usepackage[utf8]{inputenc}
\usepackage[T1]{fontenc}
\usepackage{lmodern} \normalfont %to load T1lmr.fd 
\DeclareFontShape{T1}{lmr}{bx}{sc} { <-> ssub * cmr/bx/sc }{}
\usepackage{pdfpages}
%\usepackage{enumitem}
\usepackage{enumerate}    
\usepackage{natbib}
\usepackage{ctable}
\usepackage{lscape}
\usepackage{array}
\usepackage{bbm}
\usepackage{hyperref}
\usepackage{graphicx}
\usepackage{epsf}
%\usepackage{fancyhdr}
\usepackage{setspace}
\usepackage{verbatim}
%\usepackage{courier}
\usepackage{color}
\usepackage{hyperref}
\usepackage{fncylab}
\usepackage{authblk}
\usepackage{textcomp}
%\usepackage{tmmath} %_   (TM-Math/Times)
%\usepackage{cmbright}
%\usepackage{ccfonts} \usepackage[T1]{fontenc}
%\usepackage[garamond]{mathdesign}
\usepackage{xcolor}
\usepackage[normalem]{ulem}
    

%\usepackage{newpxmath}
%\usepackage{newpxtext}
%\usepackage{sectsty}
\usepackage{lipsum}
\definecolor{ugablue}{cmyk}{1.00,.62,.0,.52}
%\definecolor{ugablue}{rgb}{0.1,0.2,0.65}
\definecolor{ugagrey}{cmyk}{.42,.08,.0,.40}
%\hypersetup{
  %colorlinks   = true, %Colours links instead of ugly boxes
  %urlcolor     = ugablue, %Colour for external hyperlinks
  %linkcolor    = DarkBlue, %Colour of internal links
 % citecolor   = DarkBlue %Colour of citations
%}
\hypersetup{
  colorlinks   = true, %Colours links instead of ugly boxes
  urlcolor     = ugablue, %Colour for external hyperlinks
  linkcolor    = MidnightBlue, %Colour of internal links
  citecolor   = MidnightBlue %Colour of citations
}
\newtheorem{theoreme}{Théorème}[section]
\newtheorem*{theoreme*}{Théorème}
%\newtheorem{conjecture}{Conjecture}[section]
%\newtheorem{corollary}{Corollary}[section]
\newtheorem{lemme}{Lemme}[section]
\newtheorem{proposition}{Proposition}[section]
%\newtheorem{propriete}{Propri\'et\'e}
%\newtheorem{assumption}{Assumption}
%\renewcommand{\theassumption}{A\arabic{assumption}}
%\newtheorem{hypothese}{Hypothèse}
%\renewcommand{\thehypothese}{H\arabic{hypothese}}
%\newtheorem{condition}{Condition}
%\renewcommand{\thecondition}{C\arabic{condition}}
%\newtheorem{propriete}{Propriété}
%\renewcommand{\thepropriete}{P\arabic{propriete}}
%\newtheorem{remarque}{Remarque}
%\theoremstyle{definition}
%\newtheorem{definition}{Définition}
%\labelformat{definition}{D\arabic{definition}}
%\newtheorem{exemple}{Exemple}
%\renewcommand{\thedefinition}{Définition D\arabic{definition}}
\newcommand{\argmax}{\operatornamewithlimits{arg\,max\,}}
\newcommand{\argmin}{\operatornamewithlimits{arg\,min\,}}
\providecommand{\plim}{\underset{n \rightarrow \infty}{\operatornamewithlimits{plim}}}
\def\inprobLOW{\rightarrow_p}
\def\inprobHIGH{\,{\buildrel p \over \rightarrow}\,} 
\def\inprob{\,{\inprobHIGH}\,} 
\def\indist{\,{\buildrel d \over \rightarrow}\,} 
\def\F{\mathbb{F}}
\def\R{\mathbb{R}}
\def\N{\mathbb{N}}
\newcommand{\gmatrix}[1]{\begin{pmatrix} {#1}_{11} & \cdots &
    {#1}_{1n} \\ \vdots & \ddots & \vdots \\ {#1}_{m1} & \cdots &
    {#1}_{mn} \end{pmatrix}}
\newcommand{\iprod}[2]{\left\langle {#1} , {#2} \right\rangle}
\newcommand{\norm}[1]{\left\Vert {#1} \right\Vert}
\newcommand{\abs}[1]{\left\vert {#1} \right\vert}
\renewcommand{\det}{\mathrm{det}}
%\newcommand{\rank}{\mathrm{rank}}
\newcommand{\spn}{\mathrm{span}}
\newcommand{\row}{\mathrm{Row}}
\newcommand{\col}{\mathrm{Col}}
\renewcommand{\dim}{\mathrm{dim}}
\newcommand{\prefeq}{\succeq}
\newcommand{\pref}{\succ}
\newcommand{\seq}[1]{\{{#1}_n \}_{n=1}^\infty }
%\providecommand{\limp}{\underset{n \rightarrow \infty}{\overset{p}{\longrightarrow}}}
%\providecommand{\limp}{\underset{n \rightarrow +\infty}{\overset{p.}{\longrightarrow}}}
%\providecommand{\limp}{\overset{p.}{\longrightarrow}}
\renewcommand{\to}{\rightarrow}
\providecommand{\limp}{\overset{p}{\rightarrow}}
%\providecommand{\limp}{\overset{p}{\to}}
\providecommand{\limps}{\overset{p.s.}{\rightarrow}}
%\providecommand{\limps}{\overset{p.s.}{\to}}
\providecommand{\limd}{\overset{d}{\rightarrow}}
\providecommand{\liml}{\overset{L}{\rightarrow}}
\providecommand{\limlp}{\overset{L^p}{\rightarrow}}
%\providecommand{\limd}{\overset{d}{\to}}
%\providecommand{\limps}{\underset{n \rightarrow +\infty}{\overset{p. s.}{\longrightarrow}}}
%\providecommand{\limp}{\overset{p}{\longrightarrow}}
%\providecommand{\limd}{\underset{n \rightarrow +\infty}{\overset{d.}{\longrightarrow}}}
\providecommand{\approxdist}{\overset{\mathrm{a}}{\sim}}
\providecommand{\sima}{\overset{\mathrm{a}}{\sim}}
%\providecommand{\limd}{\overset{d.}{\longrightarrow}}
%\newcommand{\sumobs}{\underset{i=1}{\overset{n}{\sum}}}
\providecommand{\limprob}{\overset{p}{\underset{N\to +\infty}{\longrightarrow}}}
\providecommand{\limloi}{\overset{L}{\underset{N\to +\infty}{\longrightarrow}}}
\providecommand{\limpsure}{\overset{p.s.}{\underset{N\to +\infty}{\longrightarrow}}}
\newcommand{\sumobs}{\sum_{i=1}^n}
\newcommand{\sumiton}{\sum_{i=1}^n}
\newcommand{\sumi}{\underset{i=1}{\overset{n}{\sum}}}
\newcommand{\prodi}{\underset{i=1}{\overset{n}{\prod}}}
\newcommand{\sumt}{\underset{t=1}{\overset{T}{\sum}}}
\newcommand{\sumti}{\underset{t=1}{\overset{T_i}{\sum}}}
\newcommand{\sumitok}{\underset{i=1}{\overset{k}{\sum}}}
\newcommand{\sumjton}{\underset{j=1}{\overset{n}{\sum}}}
\newcommand{\sumttoT}{\underset{t=1}{\overset{T}{\sum}}}
\newcommand{\sumkton}{\underset{k=1}{\overset{n}{\sum}}}
\newcommand{\sumjtoinfty}{\underset{j=1}{\overset{\infty}{\sum}}}
\newcommand{\sumjzerotoinfty}{\underset{j=0}{\overset{\infty}{\sum}}}
\newcommand{\sumkzerotoinfty}{\underset{k=0}{\overset{\infty}{\sum}}}
\newcommand{\sumkonetoinfty}{\underset{k=1}{\overset{\infty}{\sum}}}
\newcommand{\sumlzerotoinfty}{\underset{l=0}{\overset{\infty}{\sum}}}
\newcommand{\summinftoinf}{\underset{-\infty}{\overset{\infty}{\sum}}}
\newcommand{\limntoinfty}{\underset{n\rightarrow\infty}{\lim}}
\newcommand{\prodobs}{\underset{i=1}{\overset{n}{\prod}}}
\newcommand{\sumtoinf}{\underset{i=0}{\overset{\infty}{\sum}}}
\providecommand{\sumiN}{{\sum_{i=1}^N}}
\providecommand{\sumtT}{{\sum_{t=1}^T}}
%\renewcommand{\to}{{\rightarrow}}
\providecommand{\Lsq}{{\mathcal{L}^2}}
\providecommand{\En}{\mathbb{E}_n}
\providecommand{\Gn}{\mathrm{G}_n}
%\providecommand{\Var}{\mathbf{Var}}
%\providecommand{\Var}{\mathsf{Var}}
%\providecommand{\Var}{\mathrm{Var}}
%\providecommand{\Vr}{\mathbb{V}}
\providecommand{\Er}{\mathbb{E}}
%\providecommand{\Exp}{\mathbf{E}}
%\providecommand{\Exp}{\mathsf{E}}
%\providecommand{\Exp}{\mathbb{E}}
%\providecommand{\Exp}{\mathrm{E}}
%\DeclareMathOperator{\Exp}{\mathrm{E}}
%\DeclareMathOperator{\Var}{\mathrm{Var}}
%\DeclareMathOperator{\Cov}{\mathrm{Cov}}
%\DeclareMathOperator{\corr}{\mathrm{corr}}
\DeclareMathOperator{\indfun}{\mathbf{1}}
\DeclareMathOperator{\Exp}{E}
\DeclareMathOperator{\Expbar}{\bar{E}}
\DeclareMathOperator{\Expn}{\mathbb{E}_n}
\DeclareMathOperator{\Var}{Var}
\DeclareMathOperator{\Vr}{V}
\DeclareMathOperator{\Avar}{AVar}
\DeclareMathOperator{\Cov}{Cov}
\DeclareMathOperator{\Cv}{Cv}
\DeclareMathOperator{\corr}{corr}
\providecommand{\Cr}{\mathcal{C}}
\DeclareMathOperator{\perps}{\perp_s}
%\DeclareMathOperator{\Prob}{Pr}
\DeclareMathOperator{\Prob}{P}
\DeclareMathOperator{\prob}{p}
\DeclareMathOperator{\loss}{L}
%\DeclareMathOperator{\Exp}{\mathbf{E}}
%\DeclareMathOperator{\Var}{\mathbf{Var}}
%\DeclareMathOperator{\Cov}{\mathbf{Cov}}
%\DeclareMathOperator{\corr}{\mathbf{corr}}
%\DeclareMathOperator{\Prob}{\mathbf{Pr}}
%\providecommand{\Cov}{\mathbf{Cov}}
%\providecommand{\Cov}{\mathsf{Cov}}
%\providecommand{\Cov}{\mathrm{Cov}}
%\providecommand{\corr}{\mathbf{corr}}
\providecommand{\Id}{\mathbf{I}}
\providecommand{\Ind}{\mathbf{1}}
\providecommand{\uvec}{\mathbf{1}}
\providecommand{\vecOnes}{\mathbf{1}}
%\providecommand{\vecOnes}{\mathbf{e}}
\providecommand{\Rang}{\mathrm{Rang}}
\providecommand{\Trace}{\mathrm{Trace}}
%\providecommand{\Tr}{\mathrm{Tr}}
%\providecommand{\Cov}{\mathbb{C}\mathrm{ov}}
\providecommand{\Diag}{\mathrm{Diag}}
\providecommand{\Pred}{\mathcal{P}}
\providecommand{\sp}{\mathrm{span}}
\providecommand{\CI}{\mathrm{CI}}
\providecommand{\reg}{\mathrm{r}}
\providecommand{\Likelihood}{\mathrm{L}}
%\renewcommand{\Pr}{{\mathbb{P}}}
\providecommand{\set}[1]{\left\{#1\right\}}
\providecommand{\asyvar}{\mathrm{AsyVar}}
\providecommand{\dapprox}{\overset{a.}{\approx}}
%\providecommand{\plim}{\operatornamewithlimits{plim}}
\newcommand\indep{\protect\mathpalette{\protect\independenT}{\perp}}
%\newcommand{\independent}{\perp \!\!\! \perp}
\DeclareMathOperator{\indpnt}{\perp \!\!\! \perp}
\def\independenT#1#2{\mathrel{\setbox0\hbox{$#1#2$}%
  \copy0\kern-\wd0\mkern4mu\box0}} 
\providecommand{\iiddat}{\{(y_i, x_i)\}_{i=1}^n}
\newcommand{\dsim}{\stackrel {\textrm{appr }} {\sim} }
%\providecommand{\dsim}{{\overset{appr.}{\sim}}}
%%Commands from Econometric Theory(Slides) by J. Stachurski.
\newcommand\T{{\mathpalette\raiseT\intercal}}
\newcommand\raiseT[2]{\raisebox{0.25ex}{$#1#2$}}
\newcommand{\boldx}{ {\mathbf x} }
\newcommand{\boldu}{ {\mathbf u} }
\newcommand{\boldv}{ {\mathbf v} }
\newcommand{\boldw}{ {\mathbf w} }
\newcommand{\boldy}{ {\mathbf y} }
\newcommand{\boldb}{ {\mathbf b} }
\newcommand{\bolda}{ {\mathbf a} }
\newcommand{\boldc}{ {\mathbf c} }
\newcommand{\boldi}{ {\mathbf i} }
\newcommand{\bolde}{ {\mathbf e} }
\newcommand{\boldg}{ {\mathbf g} }
\newcommand{\boldp}{ {\mathbf p} }
\newcommand{\boldq}{ {\mathbf q} }
\newcommand{\bolds}{ {\mathbf s} }
\newcommand{\boldt}{ {\mathbf t} }
\newcommand{\boldz}{ {\mathbf z} }
\newcommand{\boldr}{ {\mathbf r} }
\newcommand{\boldd}{ {\mathbf d} }

\newcommand{\boldzero}{ {\mathbf 0} }
\newcommand{\boldone}{ {\mathbf 1} }

\newcommand{\boldalpha}{ {\boldsymbol \alpha} }
\newcommand{\boldbeta}{ {\boldsymbol \beta} }
\newcommand{\boldgamma}{ {\boldsymbol \gamma} }
\newcommand{\boldtheta}{ {\boldsymbol \theta} }
\newcommand{\boldxi}{ {\boldsymbol \xi} }
\newcommand{\boldtau}{ {\boldsymbol \tau} }
\newcommand{\boldepsilon}{ {\boldsymbol \epsilon} }
\newcommand{\boldmu}{ {\boldsymbol \mu} }
\newcommand{\boldSigma}{ {\boldsymbol \Sigma} }
\newcommand{\boldOmega}{ {\boldsymbol \Omega} }
\newcommand{\boldPhi}{ {\boldsymbol \Phi} }
\newcommand{\boldLambda}{ {\boldsymbol \Lambda} }
\newcommand{\boldGamma}{ {\boldsymbol \Gamma} }
\newcommand{\boldPi}{ {\boldsymbol \Pi} }
\newcommand{\boldpi}{ {\boldsymbol \pi} }
\newcommand{\boldphi}{ {\boldsymbol \phi} }

\newcommand{\Sigmax}{ {\boldsymbol \Sigma_{\boldx}}}
\newcommand{\Sigmau}{ {\boldsymbol \Sigma_{\boldu}}}
\newcommand{\Sigmaxinv}{ {\boldsymbol \Sigma_{\boldx}^{-1}}}
\newcommand{\Sigmav}{ {\boldsymbol \Sigma_{\boldv \boldv}}}

\newcommand{\hboldx}{ \hat {\mathbf x} }
\newcommand{\hboldy}{ \hat {\mathbf y} }
\newcommand{\hboldb}{ \hat {\mathbf b} }
\newcommand{\hboldu}{ \hat {\mathbf u} }
\newcommand{\hboldtheta}{ \hat {\boldsymbol \theta} }
\newcommand{\hboldtau}{ \hat {\boldsymbol \tau} }
\newcommand{\hboldmu}{ \hat {\boldsymbol \mu} }
\newcommand{\hboldbeta}{ \hat {\boldsymbol \beta} }
\newcommand{\hboldgamma}{ \hat {\boldsymbol \gamma} }
\newcommand{\hboldSigma}{ \hat {\boldsymbol \Sigma} }
\newcommand{\betahat}{\hat{\beta}}
\newcommand{\gammahat}{\hat{\gamma}}
\newcommand{\Ahat}{\hat{A}}
\newcommand{\epsilonhat}{\hat{\epsilon}}
\newcommand{\varepsilonhat}{\hat{\varepsilon}}
\newcommand{\sigmahat}{\hat{\sigma}}
\newcommand{\Sigmahat}{\hat{\Sigma}}
\newcommand{\boldA}{\mathbf A}
\newcommand{\boldB}{\mathbf B}
\newcommand{\boldC}{\mathbf C}
\newcommand{\boldD}{\mathbf D}
\newcommand{\boldG}{\mathbf G}
\newcommand{\boldI}{\mathbf I}
\newcommand{\boldL}{\mathbf L}
\newcommand{\boldM}{\mathbf M}
\newcommand{\boldP}{\mathbf P}
\newcommand{\boldQ}{\mathbf Q}
\newcommand{\boldR}{\mathbf R}
\newcommand{\boldX}{\mathbf X}
\newcommand{\boldU}{\mathbf U}
\newcommand{\boldV}{\mathbf V}
\newcommand{\boldW}{\mathbf W}
\newcommand{\boldY}{\mathbf Y}
\newcommand{\boldZ}{\mathbf Z}

\newcommand{\hboldU}{ \hat {\mathbf U} }
\newcommand{\hboldOmega}{ \hat{{\boldsymbol \Omega} }}
\newcommand{\bSigmaX}{ {\boldsymbol \Sigma_{\hboldbeta}} }
\newcommand{\hbSigmaX}{ \mathbf{\hat \Sigma_{\hboldbeta}} }

\newcommand{\RR}{\mathbbm R}
\newcommand{\CC}{\mathbbm C}
%\newcommand{\NN}{\mathbbm N}
\newcommand{\PP}{\mathbbm P}
\newcommand{\EE}{\mathbbm E \nobreak\hspace{.1em}}
\newcommand{\EEP}{\mathbbm E_P \nobreak\hspace{.1em}}
\newcommand{\ZZ}{\mathbbm Z}
\newcommand{\QQ}{\mathbbm Q}


\newcommand{\XX}{\mathcal X}

\newcommand{\aA}{\mathcal A}
\newcommand{\fF}{\mathscr F}
\newcommand{\bB}{\mathscr B}
\newcommand{\iI}{\mathscr I}
\newcommand{\rR}{\mathscr R}
\newcommand{\dD}{\mathcal D}
\newcommand{\lL}{\mathcal L}
\newcommand{\llL}{\mathcal{H}_{\ell}}
\newcommand{\gG}{\mathcal G}
\newcommand{\hH}{\mathcal H}
\newcommand{\nN}{\textrm{\sc n}}
\newcommand{\lN}{\textrm{\sc ln}}
\newcommand{\pP}{\mathscr P}
\newcommand{\qQ}{\mathscr Q}
\newcommand{\xX}{\mathcal X}

\newcommand{\ddD}{\mathscr D}
\newcommand{\lnorm}{\mathit{l}}


%\newcommand{\R}{{\texttt R}}
\newcommand{\risk}{\mathcal R}
\newcommand{\Remp}{R_{{\rm emp}}}

\newcommand*\diff{\mathop{}\!\mathrm{d}}
\newcommand{\ess}{ \textrm{{\sc ess}} }
\newcommand{\tss}{ \textrm{{\sc tss}} }
\newcommand{\rss}{ \textrm{{\sc rss}} }
\newcommand{\rssr}{ \textrm{{\sc rssr}} }
\newcommand{\ussr}{ \textrm{{\sc ussr}} }
\newcommand{\zdata}{\mathbf{z}_{\mathcal D}}
\newcommand{\Pdata}{P_{\mathcal D}}
\newcommand{\Pdatatheta}{P^{\mathcal D}_{\theta}}
\newcommand{\Zdata}{Z_{\mathcal D}}


\newcommand{\e}[1]{\mathbbm{E}[{#1}]}
\newcommand{\p}[1]{\mathbbm{P}({#1})}



\DeclareMathOperator{\cl}{cl}
%\DeclareMathOperator{\argmax}{argmax}
\DeclareMathOperator{\interior}{int}
%\DeclareMathOperator{\Prob}{Prob}
\DeclareMathOperator{\kernel}{ker}
\DeclareMathOperator{\diag}{diag}
\DeclareMathOperator{\lag}{\mathbf{L}}
\DeclareMathOperator{\AR}{\mathrm{AR}}
\DeclareMathOperator{\MA}{\mathrm{MA}}
\DeclareMathOperator{\ARMA}{\mathrm{ARMA}}
\DeclareMathOperator{\sgn}{sgn}
\DeclareMathOperator{\determinant}{det}
\DeclareMathOperator{\vech}{vech}
\DeclareMathOperator{\trace}{trace}
\DeclareMathOperator{\Tr}{Tr}
\DeclareMathOperator{\Span}{span}
\DeclareMathOperator{\rank}{rank}
%\DeclareMathOperator{\cov}{cov}
%\DeclareMathOperator{\corr}{corr}
\DeclareMathOperator{\range}{rng}
%\DeclareMathOperator{\Var}{Var}
\DeclareMathOperator{\mse}{mse}
\DeclareMathOperator{\se}{se}
%\DeclareMathOperator{\row}{row}
%\DeclareMathOperator{\col}{col}
\DeclareMathOperator{\dimension}{dim}
\DeclareMathOperator{\fracpart}{frac}
\DeclareMathOperator{\proj}{proj}
\DeclareMathOperator{\colspace}{colspace}
%\theoremstyle{plain}
%\newtheorem{axiom}{Axiom}[section]
%\newtheorem{theorem}{Theorem}[section]
%\newtheorem{corollary}{Corollary}[section]
%\newtheorem{lemma}{Lemma}[section]
%\newtheorem{proposition}{Proposition}[section]
%
\theoremstyle{definition}
%\newtheorem{definition}{Definition}[section]
%\newtheorem{example}{Example}[section]
%\newtheorem{remark}{Remark}[section]
%\newtheorem{notation}{Notation}[section]
\newtheorem{assumption}{Assumption}
\renewcommand{\theassumption}{A\arabic{assumption}}
%\newtheorem{condition}{Condition}[section]
%\newtheorem{exercise}{Ex.}[section]
%\newtheorem{fact}{Fact}[section]
%%
\newtheoremstyle{interpret}% name
{12pt}% Space above
{5pt}% Space below
{}% Body font
{}% Indent amount
{\bfseries}% Theorem head font
{:}% Punctuation after theorem head
{.3em}% Space after theorem head
{}% Theorem head spec (can be left empty, meaning `normal')

\theoremstyle{interpret}
\newtheorem{interpret}{\textsc{Interprétation}}
\renewcommand{\theinterpret}{}

%\newtheoremstyle{interpretation}% name
%{12pt}% Space above
%{5pt}% Space below
%{}% Body font
%{}% Indent amount
%{\bfseries}% Theorem head font
%{}% Punctuation after theorem head
%{.5em}% Space after theorem head
%{}% Theorem head spec (can be left empty, meaning `normal')


\theoremstyle{remark}
\newtheorem{interpretation}{Interprétation}
\newtheorem*{interpretation*}{Interprétation}

\theoremstyle{remark}
%\newtheorem{remarque}{Remarque}%[section]
%\newtheorem*{remarque*}{Remarque}
\newtheorem{remark}{Remark}%[section]
\newtheorem*{remark*}{Remark}
%\newtheorem{comment}{Comment}%[section]
%\newtheorem*{comment*}{Comment
%\usepackage{mdframed}
%\newtheorem{step}{Step}[section]
%\newtheorem{rem}{Comment}[section]
%\newtheorem{ex}{Example}[section]
%\newtheorem{hist}{History}[section]
%\newtheorem*{ex*}{Example}
\theoremstyle{plain}
\newtheorem{propriete}{Propri\'et\'e}%[section]
\renewcommand{\thepropriete}{P\arabic{propriete}}
\newtheorem{lemma}{Lemma}
\newtheorem*{lemma*}{Lemma}
\newtheorem{theorem}{Theorem}
\newtheorem*{theorem*}{Theorem}
\theoremstyle{definition}
\newtheorem{definition}{Définition}%[section]
\theoremstyle{definition}

%\newtheorem{definitionx}{D\'efinition}%[section]
%\newenvironment{definition}
 %{\pushQED{\qed}\renewcommand{\qedsymbol}{$\triangle$}\definitionx}
 %{\popQED\enddefinitionx}

\newtheorem{condition}{Condition}
\renewcommand{\thecondition}{C\arabic{condition}}
\theoremstyle{remark}
%\newtheorem{exemple}{Exemple}
%\newtheorem*{exemple*}{Exemple}
\newtheorem{example}{Example}
\newtheorem*{example*}{Example}
\newtheorem{remarque}{Remarque}
\newtheorem*{remarque*}{Remarque}
\newtheorem{exemple}{Exemple}
\newtheorem*{exemple*}{Exemple}
%\newtheorem{example}{Example}[section]
%\newtheorem{remark}{Remark}[section]
%\newtheorem{notation}{Notation}[section]
%\newtheorem{assumption}{Assumption}[section]
%\newtheorem{condition}{Condition}[section]
%\newtheorem{exercise}{Ex.}[section]
%\newtheorem{fact}{Fact}[section]
\newenvironment{prt}
  {
  \begin{mdframed}
  \begin{propriete}}
  {\end{propriete}
  \end{mdframed}
  }
  
  \newenvironment{dft}
  {
  \begin{mdframed}\begin{definition}}
  {\end{definition}\end{mdframed}
  }



%\newenvironment{example*}
 % {\begin{mdframed}\begin{ex*}}
  %{\end{ex*}\end{mdframed}}


\usepackage{xcolor}
 \colorlet{linkequation}{ugablue} 
 
\newcommand*{\SavedEqref}{}
 \let\SavedEqref\eqref 
\renewcommand*{\eqref}[1]{%
\begingroup \hypersetup{
      linkcolor=linkequation,
linkbordercolor=linkequation, }%
\SavedEqref{#1}%
 \endgroup
}

\newcommand*{\refeq}[1]{%
 \begingroup
\hypersetup{ 
linkcolor=linkequation, 
linkbordercolor=linkequation,
}%
\ref{#1}%
 \endgroup
}

%%Nouvelle commande pour symbole de transposition: 11022024.
\makeatletter
\newcommand*{\tr}{%
  {\mathpalette\@tr{}}%
}
\newcommand*{\@tr}[2]{%
  % #1: math style
  % #2: unused
  \raisebox{\depth}{$\m@th#1\intercal$}%
}
\makeatother
%%

\newcommand{\bxi}{ \boldx_i }
\usepackage{tikz}
\usepackage{enumitem}
\usepackage{txfonts}
\usepackage{mathrsfs}
\usepackage{fancyhdr}

\usepackage{fancyhdr}
\usetikzlibrary{calc,decorations.pathmorphing,patterns}
\usepackage{listofitems}
\tikzstyle{mynode}=[thick,draw=blue,fill=blue!20,circle,minimum size=22]
\usepackage{amsaddr}
\usepackage{mathrsfs}

\makeatletter

\pgfdeclaredecoration{penciline}{initial}{
    \state{initial}[width=+\pgfdecoratedinputsegmentremainingdistance,auto corner on length=1mm,]{
        \pgfpathcurveto%
        {% From
            \pgfqpoint{\pgfdecoratedinputsegmentremainingdistance}
                            {\pgfdecorationsegmentamplitude}
        }
        {%  Control 1
        \pgfmathrand
        \pgfpointadd{\pgfqpoint{\pgfdecoratedinputsegmentremainingdistance}{0pt}}
                        {\pgfqpoint{-\pgfdecorationsegmentaspect\pgfdecoratedinputsegmentremainingdistance}%
                                        {\pgfmathresult\pgfdecorationsegmentamplitude}
                        }
        }
        {%TO 
        \pgfpointadd{\pgfpointdecoratedinputsegmentlast}{\pgfpoint{1pt}{1pt}}
        }
    }
    \state{final}{}
}
\makeatother

\pagestyle{fancy}
%\renewcommand{\subsection{mark}[1]{\markright{#1}{}}
\fancyhead{}
\fancyfoot{} 
%\fancyhead[LE,LO]{\tiny{\thepage}}
\fancyhead[C]{\small\textsc{Économètrie: L3 MIASH, S2}}
%fancyhead[CE,CO]{\tiny{\rightmark}}
\fancyhead[L]{\small\textsc{Université de Grenoble Alpes}}
\fancyfoot[C]{\small{\thepage}}
%\fancyfoot[R]{\small \textcopyright \ \  \small\textsc
\fancyhead[R]{ \small\textsc{M. W. Urdanivia}}
%\renewcommand{\headrulewidth}{0pt}
\renewcommand{\footrulewidth}{0pt}

%\pagenumbering{roman}


\begin{document} 
\usetikzlibrary{positioning}
\usetikzlibrary{snakes}
\usetikzlibrary{calc}
\usetikzlibrary{arrows}
\usetikzlibrary{decorations.markings}
\usetikzlibrary{shapes.misc}
\usetikzlibrary{shapes}
%\tikzset{block/.style={draw, rectangle, fill=gray!20, 
%\tikzset{empty/.style={draw, rectangle, fill=none, tex
%\tikzset{line/.style={draw, -latex'}}
%\onehalfspace

%\includepdf{trame}


\begin{titlepage}
\centering
	%\includegraphics[width=0.15\textwidth]{logoUGA2020}
	{\scshape\Large \textsc{Université de Grenoble Alpes\\(L3 MIASH, S2)}\par}
	\vspace{0.5cm}
	{\Large\bfseries \scshape\Large \textsc{ÉCONOMÉTRIE}\par}
	%{\scshape\large \textsc{Extremum Estimators(1)}\par}
	%\vspace{1cm}
	\vspace{0.5cm}
	{\Large\bfseries \textsc{Régression Linéaire et Hétéroscédasticité:
    } \par}
    \vspace{1cm}   
	{\Large\bfseries \textsc{Moindres Carrés Généralisés} \par}
	\vspace{1cm}
	{(\textsc{Cette version: \today})\par}
	\vspace{1cm}
	{\large \textsc{Michal W. Urdanivia}
	\footnote{Contact:  
	\href{mailto:michal.wong-urdanivia@univ-grenoble-alpes.fr}{michal.wong-urdanivia@univ-grenoble-alpes.fr}, 
	 Université de Grenoble Alpes,  Faculté d'\'Economie, GAEL.}\par}
	 %\includegraphics[width=0.15\textwidth]{logoUGA}
	%\vfill
	%supervised by\par
	%Dr.~Mark \textsc{Brown}
%\vfill
% Bottom of the page
	
\end{titlepage}


\newpage

\tableofcontents

\newpage

\newpage

\tableofcontents

\newpage

\section{Estimateur des moindres carrés généralisés}

On considère le modèle linéaire sans endogénéité étudié dans les cours
précédents. Toutefois, plutôt que la condition $\Exp(X_iU_i) = 0$ nous
allons utiliser la condition(plus forte) $\Exp(U_i|X_i)=0$. En résumé,
le conditions que l'on impose sont les suivantes;

\begin{condition}
Les données  $\{(Y_i, X_i), i = 1,\ldots,n\}$ sont un échantillon i.i.d.
\label{cond1}
\end{condition}
\begin{condition} $Y_i$  et $X_i$ vérifient,
\begin{align*} 
Y_i= X_i^\tr\beta + U_i \ \ i = 1,\ldots,n
\end{align*}
où $U_i$ est une variable inobservée(ou terme d'erreur) vérifiant $\Exp(U_i) = 0$.
\label{cond2} 
\end{condition}
\begin{condition}$X_i$ est exogène par rapport à $U_i$,
\begin{align*}
\Exp(U_i|X_i) = 0
\end{align*}
\label{cond3} 
\end{condition}
\begin{condition} 
La matrice $\Exp(X_iX_i^\tr)$ est finie et définie positive.
\label{cond4}
\end{condition}
\begin{condition}
$\Exp(X_{i,k}^4) < \infty$ , pour tout $k=1,\ldots,K$.
\label{cond5}
\end{condition}
\begin{condition}
$\Exp(U_i^4) < \infty$
\label{cond6}
\end{condition}
\begin{condition}
$\Exp(U_i^2X_iX_i^\tr)$ est définie positive.
\label{cond7}
\end{condition}

Comme cela a été déjà noté la condition \eqref{cond3} est plus forte que
celle dont on a besoin pour établir la convergence et la normalité asymptotique de l'estimateur des
MCO. Toutefois elle va nous permettre d'étudier le problème de
l'efficacité dans le cas d'erreurs
hétécédastiques où $\Exp(U_i^2|X_i)=\sigma^2_i$, et $\sigma^2_i$ étant une
fonction de $X_i$: $\sigma^2_i =\sigma^2(X_i)$.

\begin{exemple}
Supposons que $Y_{i,j}=X_{i,j}^\tr\beta + U_{i,j}$ pour
$i=1,...,n$($n$ secteurs industriels) et $j=1,...,m_i$($m_i$
entreprises dans le secteur $i$). Supposons les observations
i.i.d. selon $i$ et $j$, et que l'on observe seulement des valeurs
moyennes des variables pour les $n$ secteurs, soit: $\bar{Y}_i =
m_i^{-1}\underset{j=1}{\overset{m_i}{\sum}} Y_{i,j}$, et $\bar{X}_i=
m_i^{-1}\underset{j=1}{\overset{m_i}{\sum}} X_{i,j}$. Supposons aussi
que des erreurs $U_{i,j}$ homoscédastiques,
i.e.,$\Exp(U_{i,j}^2|X_{i,j}) = \sigma^2$ pour tout $i=1,...,n$ et
$j=1,...,m_i$. Cependant,
$\bar{U}_i=m_i^{-1}\underset{j=1}{\overset{m_i}{\sum}} U_{i,j}$,
et $\Exp\left(\bar{U}_i^2|\bar{X}_i\right) = \sigma^2/m_i$.
\end{exemple}
En présence d'hétéroscédasticité, l'estimateur des MCO est convergent
et asymptotiquement normal, mais il n'est plus efficace. Il existe en
effet un estimateur avec une variance asymptotique plus petite. Sous
\eqref{cond1}, et \eqref{cond3}, $\Exp\left(\hat{\beta}_n|\mathbf{X}\right)
= \beta$(sans biais), et,
\begin{align*}
\Vr\left(\hat{\beta}_n|\mathbf{X}\right) &=
                                               (\mathbf{X}^\tr\mathbf{X})^{-1}
\mathbf{X}^\tr\mathbf{D}\mathbf{X}(\mathbf{X}^\tr\mathbf{X})^{-1}
\end{align*}
avec,
\begin{align*}
\mathbf{D} &=
\left(
\begin{array}{cccccc}
\sigma^2_1&0&.&.&.&0\\
0&\sigma^2_2&.&.&.&0\\
.&0&.&.&.&.\\
.&.&.&.&.&.\\
.&.&.&.&.&.\\
0&.&.&.&0&\sigma^2_n\\
\end{array}
\right)
\end{align*}
Supposons aussi que $\sigma_i^2 = \sigma^2(X_i)$ soit connu pour tout
$i$. L'estimateur des moindres carrés généralisés(MCG) est défini
comme suit,

\begin{align}
\hat{\beta}_n^{MCG} &=
                          \left(\mathbf{X}^\tr\mathbf{D}^{-1}\mathbf{X}\right)^{-1}\mathbf{X}^\tr\mathbf{D}^{-1}\mathbf{Y}
  \nonumber\\
&= \sumobs(\sigma^{-2}_iX_iX_i^\tr)^{-1}\sumobs \sigma^{-2}X_iY_i
\label{eqGLS1}
\end{align}
Dans le cas de $\mathbf{D}$ diagonale, l'estimateur des MCG est aussi
appelé \emph{estimateur des moindres carrés pondérés} car il fait
intervenir les moyennes pondérées des termes $X_iX_i^\tr$ et $X_iY_i$
pondérées par les termes $\sigma^{-2}_i$.\\
Sous les conditions \eqref{cond1} et \eqref{cond3}, on établit que
$\hat{\beta}_n^{MCG}$ est sans biais,
\begin{align*}
\Exp\left(\hat{\beta}_n^{MCG} | \mathbf{X}\right) &= \beta +
                                                       \left(\mathbf{X}^\tr\mathbf{D}^{-1}\mathbf{X}\right)^{-1}\mathbf{X}^\tr\mathbf{D}^{-1}\Exp(\mathbf{U}|\mathbf{X})\\
&=\beta
\end{align*}

Et sa variance est donnée par,

\begin{align*}
\Vr\left(\hat{\beta}_n^{MCG}|\mathbf{X}\right)&=
                                                    \left(\mathbf{X}^\tr\mathbf{D}^{-1}\mathbf{X}\right)^{-1}\mathbf{X}^\tr\mathbf{D}^{-1}\Exp\left(\mathbf{U}\mathbf{U}^\tr|\mathbf{X}\right)\mathbf{D}^{-1}\mathbf{X}
                                                    \left(\mathbf{X}^\tr\mathbf{D}^{-1}\mathbf{X}\right)^{-1}\\
&= \left(\mathbf{X}^\tr\mathbf{D}^{-1}\mathbf{X}\right)^{-1}
\end{align*}
Montrons à présent que
$\Vr\left(\hat{\beta}_n^{MCG}|\mathbf{X}\right)\leq
\Vr\left(\hat{\beta}_n|\mathbf{X}\right)$. Notons d'abord que,

\begin{align*}
\Vr\left(\hat{\beta}_n^{MCG}|\mathbf{X}\right)\leq
\Vr\left(\hat{\beta}_n|\mathbf{X}\right) &\Leftrightarrow
\left(\Vr\left(\hat{\beta}_n^{MCG}|\mathbf{X}\right)\right)^{-1}\geq
\left(\Vr\left(\hat{\beta}_n|\mathbf{X}\right)\right)^{-1}
\end{align*}
Ensuite que,
\begin{align*}
\mathbf{X}^\tr\mathbf{D}^{-1}\mathbf{X} -
  \mathbf{X}^\tr\mathbf{X}\left(\mathbf{X}^\tr\mathbf{D}\mathbf{X}\right)^{-1}
  \mathbf{X}^\tr\mathbf{X} &=
                              \mathbf{X}^\tr\mathbf{D}^{-1/2}\left(\Id
                              - \mathbf{D}^{1/2}\mathbf{X}\left(\mathbf{X}^\tr\mathbf{D}\mathbf{X}\right)^{-1}\mathbf{X}\mathbf{D}^{1/2}\right)\mathbf{D}^{-1/2}\mathbf{X}
\end{align*}
où l'on note que $\Id
                              -
                              \mathbf{D}^{1/2}\mathbf{X}\left(\mathbf{X}^\tr\mathbf{D}\mathbf{X}\right)^{-1}\mathbf{X}\mathbf{D}^{1/2}$
                              est symmétrique et semi-définie
                              positive, et par conséquent,

\begin{align*}
\mathbf{X}^\tr\mathbf{D}^{-1}\mathbf{X} &\geq
  \mathbf{X}^\tr\mathbf{X}\left(\mathbf{X}^\tr\mathbf{D}\mathbf{X}\right)^{-1}
  \mathbf{X}^\tr\mathbf{X}
\end{align*}

L'efficacité de l'estimateur des MCG résulte de l'application du
théorème de Gauss-Markov. En effet, avec l'hétéroscédasticité des
erreurs une des conditions pour l'obtenir n'est plus satisfaite. Mais
considèrons le modèle tranformé suivant,
\begin{align*}
\tilde{Y}_i &=\tilde{X}_i^\tr\beta + \tilde{U}_i
\end{align*}
où $\tilde{Y}_i = Y_i/\sigma$, $\tilde{X}_i = X_i/\sigma$, et
$\tilde{U}_i = U_i/\sigma$. Les erreurs transformés sont
homoscédastiques,
\begin{align*}
\Exp\left(\tilde{U}_i^2|X_i\right) &= \sigma_i^2/\sigma_i^2\\
&=1
\end{align*}
et par le théorème de Gauss-Markov l'estimateur BLUE est donné par,
\begin{align*}
\left(\sumobs\tilde{X}_i\tilde{X}_i^\tr\right)^{-1}\sumobs
  \tilde{X}_i\tilde{Y}_i&=
\left(\sumobs\sigma^{-2}_i X_iX_i^\tr\right)^{-1}\sumobs
  \sigma^{-2}_iX_iY_i\\
&=\hat{\beta}_n^{MCG}
\end{align*}

\section{Propriétés asymptotiques}

On commence par discuter la convergence de l'estimateur. Pour cela
écrivons,
\begin{align*}
\hat{\beta}_n^{MCG} &= \beta + \left(\sumobs (\sigma^{-2}_iX_iX_i^\tr\right)^{-1}\sumobs\sigma^{-2}_iX_iU_i
\end{align*}
Nous supposons que la fonction $\sigma^2(X_i)$ présente une borne
inférieure, i.e., avec une probabilité de 1,
$\sigma^2(X_i)>\underline{\sigma}^2>0$. Ce qui garantit, que
$\Exp(\sigma^{-2}_iX_iX_i^\tr)$ "n'explose pas". Pour $k, l =
1,...,K$, nous avons
$\Exp\left(\abs{\sigma_i^{-2}X_{i,k}X_{i,l}}\right) \leq
\underline{\sigma}^2\Exp\left(\abs{X_{i,k}X_{i,l}}\right)<\infty$(en
utilisant la condition \eqref{cond4}). Par la loi faible des grands
nombres et le théorème de Slutsky,

\begin{align*}
 \left(\sumobs \sigma^{-2}_iX_iX_i^\tr\right)^{-1} &\limp \left(\Exp(\sigma_i^{-2}X_iX_i^\tr)\right)^{-1}
\end{align*}

Nous avons aussi que,

\begin{align*}
\Exp\left(\sigma_i^{-2}X_iU_i\right) &= \Exp\left(\sigma_i^{-2}
                                      X_i\Exp(U_i|X_i)\right)\\
&=0
\end{align*}
et ainsi,
\begin{align*}
\sumobs\sigma^{-2}_iX_iU_i&\limp 0
\end{align*}
Par conséquent, $\hat{\beta}_n^{MCG}\limp \beta$ lorsque $n
\rightarrow \infty$. Notons qu'en général,  $\hat{\beta}_n^{MCG}$
  n'est pas convergent sous la condition $\Exp(X_iU_i)=0$, étant donné
  que comme $\sigma_i^2$ est une fonction de $X_i$ on ne peut pas
  garantir que $\Exp\left(\sigma_i^{-2}X_iU_i\right) = 0$ avec
  $\Exp(X_iU_i)=0$. Pour établir la normalité asymptotique de
  $\hat{\beta}_n^{MCG}$, écrivons,
\begin{align*}
n^{1/2}\left(\hat{\beta}_n^{MCG} -
  \beta\right)&=\left(n^{-1}\sumobs
                \sigma_i^{-2}X_iX_i^\tr\right)^{-1}n^{-1/2} \sumobs\sigma_i^{-2}X_iU_i
\end{align*}
Nous avons,
\begin{align*}
\Vr\left(\sigma_i^{-2}X_iU_i\right)&=
                                     \Exp\left(\sigma_i^{-4}X_iX_i^\tr
                                     U_i^2\right)\\
&= \Exp\left(\sigma_i^{-4}X_iX_i^\tr
                                     \Exp(U_i^2|X_i)\right)\\
& = \Exp\left(\sigma_i^{-2}X_iX_i^\tr\right)
 \end{align*}
et par conséquent,
\begin{align*}
n^{1/2}\left(\hat{\beta}_n^{MCG} -
  \beta\right)&\limd
                \left(\Exp\left(\sigma_i^{-2}X_iX_i^\tr\right)\right)^{-1}\mathcal{N}\left(0,
                \Exp\left(\sigma_i^{-2}X_iX_i^\tr\right)\right)\\
&=\mathcal{N}\left(0,
                \Exp\left(\sigma_i^{-2}X_iX_i^\tr\right)^{-1}\right)
\end{align*}

\section{Moindres carrés quasi-généralisés}
L'estimateur des MCG n'est pas applicable dès lors que l'on ne connaît
pas $\sigma_i^2$. Une solution naturelle à ce problème consiste à
remplacer le terme inconnu $\sigma_i^2$ dans \eqref{eqGLS1} par un
estimateur $\hat{\sigma}^2_i$. Pour cela une pratique courante
consiste à supposer que,

\begin{align}
\sigma_i^2&= Z_i^\tr\alpha
\label{eqGLS2}
\end{align}

où $Z_i$ est un vecteur $q\times 1$ de fonctions de $X_i$. Ils
consistent en général en produits et produits croisés des éléments de
$X_i$. \'Etant donné
que $\sigma_i^2 = \Exp(U_i^2|X_i)$, on peut écrire,

\begin{align*}
U_i^2&=Z_i^\tr\alpha + \nu_i
\end{align*}

où $\Exp(\nu_i|Z_i)=0$. Ce modèle est appelé \emph{régression
  scédastique}. Comme les termes $U_i$ ne sont pas observés on utilise
les résidus de la régression estimée par MCO(de $Y_i$ sur $X_i$),
$\hat{U}_i$, et on estime $\alpha$ par,

\begin{align*}
\hat{\alpha}_n &= \left(\sumobs Z_iZ_i^\tr\right)^{-1}\sumobs Z_i\hat{U}_i^2
\end{align*}

Et l'on peut montrer que $\hat{\alpha}_n\limp \alpha$ et
$n^{1/2}\left(\hat{\alpha}_n-\alpha\right)\limd\mathcal{N}\left(0,
\Vr_\alpha\right)$ où $\Vr_\alpha$ est la même variance que dans le
cas où les termes $U_i^2$ étaient observés. L'estimateur des moindres
carrés quasi-généralisés(MCQG) est alors donné par,

\begin{align*}
\hat{\beta}_n^{MCQG}
  &=\left(\mathbf{X}^\tr\hat{\mathbf{D}}_n^{-1}\mathbf{X}\right)^{-1}
\mathbf{X}^\tr\hat{\mathbf{D}}_n^{-1}\mathbf{Y}\\
&=\left(\sumobs\hat{\sigma}_i^{-2}X_iX_i^\tr\right)^{-1}\sumobs\hat{\sigma}_i^{-2}X_iY_i
\end{align*}
où
\begin{align*}
\hat{\mathbf{D}}_n &=
\left(
\begin{array}{cccccc}
\hat{\sigma}^2_1&0&.&.&.&0\\
0&\hat{\sigma}^2_2&.&.&.&0\\
.&0&.&.&.&.\\
.&.&.&.&.&.\\
.&.&.&.&.&.\\
0&.&.&.&0&\hat{\sigma}^2_n\\
\end{array}
\right)
\end{align*}
et,
\begin{align*}
\hat{\sigma}^2_i &= Z_i^\tr\hat{\alpha}_n
\end{align*}
De plus, on peut montrer que $\hat{\beta}_n^{MCQG}\limp \beta$, et
que,
\begin{align}
n^{1/2}\left(\hat{\beta}_n^{MCQG} -
  \beta\right)&\limd \mathcal{N}\left(0,
                \Exp\left(\sigma_i^{-2}X_iX_i^\tr\right)^{-1}\right)
\label{eqGLS3}
\end{align}
autrement dit, la même distribution que pour l'estimateur des MCG, à
condition que \eqref{eqGLS2} soit correctement spécifié. Ci-après nous
listons les étapes pour construire l'estimateur des MCQG.
\begin{enumerate}
\item Obtenir $\hat{\beta}_n$, l'estimateur des MCO de $\beta$.
\item Construire $\hat{U}_i = Y_i - X_i^\tr\hat{\beta}_n$.
\item Obtenir $\hat{\alpha}_n$, l'estimateur des MCO dans la
  régression de $\hat{U}_i^2$ sur $Z_i$.
\item Construire $\hat{\sigma}_i^2=Z_i^\tr\hat{\alpha}_n$.
\item Calculer $\hat{\beta}_n^{MCQG}$.
\end{enumerate}
Un problème avec l'estimateur des MCQG dans l'approche ci-avant est
que $\hat{\sigma}_i^2 = Z_i^\tr\hat{\alpha}_n$ peut être très
proche de zéro voire négatif. Plusieurs solutions sont possibles. La
première consiste à tronquer la distribution des
$\hat{\sigma}_i^2$ en choisissant $\underline{\sigma}^2 > 0$ et
posant $\hat{\sigma}_i^2 =
\max\left\{Z_i\hat{\alpha}_n,\underline{\sigma}^2 \right\}$. Une
alternative est de considérer une régression scédastique non-linéaire,
par exemple,
\begin{align*}
\sigma_i^2 &= \exp\left(Z_i^\tr\alpha\right)
\end{align*}
Dans ce cas à la troisième étape on doit régresser
$\log\left(\hat{U}_i^2\right)$ sur $Z_i$ et à la quatrième étape calculer
$\hat{\sigma}_i^2 = \exp\left(Z_i\hat{\alpha}_n\right)$.\\
La procédure pour les MCQG repose sur deux hypothèses très
fortes. D'abord que la régression scédastique soit correctement
spécifiée. En cas de mauvaise spécification, $\hat{\sigma}^2_i$
fournit seulement une approxiamtion de $\sigma_i^2$, et la variance
asymptotique dans \eqref{eqGLS3} aura la forme en " sandwich",

\[
\left(\Exp\left((Z_i^\tr\alpha)^{-1}X_iX_i^\tr\right)\right)^{-1}
\Exp\left((Z_i^\tr\alpha)^{-2}\sigma_i^2X_iX_i^\tr\right)
\left(\Exp\left((Z_i^\tr\alpha)^{-1}X_iX_i^\tr\right)\right)^{-1}
\]
et l'estimateur des MCQG aura une performance encore moin bonne que
celle de l'estimateur des MCO. En outre, si la condition
$\Exp(U_i|X_i)=0$ n'est pas satisfaite les estimateurs des MCG et des
MCQG ne seront pas convergents. Alors que l'estimateur des MCO est
moins efficace que celui des MCQG sous certaines conditions, il
fournit des estimateurs plus \emph{robustes}.

%\section{Tester l'hétéroscédasticité}
%\bibliography{../../Biblio.bib}
 \end{document}