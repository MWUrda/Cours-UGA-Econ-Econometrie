\input{../../Preambule}

\usepackage{tikz}
\usepackage{enumitem}

\usepackage{fancyhdr}
\pagestyle{fancy}
%\renewcommand{\subsection{mark}[1]{\markright{#1}{}}
\fancyhead{}
\fancyfoot{} 
%\fancyhead[LE,LO]{\tiny{\thepage}}
\fancyhead[C]{\small\textsc{Économétrie}}
%fancyhead[CE,CO]{\tiny{\rightmark}}
\fancyhead[L]{\small\textsc{UGA}}
\fancyfoot[C]{\small{\thepage}}
%\fancyfoot[R]{\small \textcopyright \ \  \small\textsc
\fancyhead[R]{ \small\textsc{M. Urdanivia}}
%\renewcommand{\headrulewidth}{0pt}
\renewcommand{\footrulewidth}{0pt}

%\pagenumbering{roman}


\begin{document} 
\usetikzlibrary{positioning}
\usetikzlibrary{snakes}
\usetikzlibrary{calc}
\usetikzlibrary{arrows}
\usetikzlibrary{decorations.markings}
\usetikzlibrary{shapes.misc}
\usetikzlibrary{shapes}
%\tikzset{block/.style={draw, rectangle, fill=gray!20, 
%\tikzset{empty/.style={draw, rectangle, fill=none, tex
%\tikzset{line/.style={draw, -latex'}}
%\onehalfspace

\begin{titlepage}
\centering
	%\includegraphics[width=0.15\textwidth]{logoUGA2020
	{\scshape\Large \textsc{Économétrie: UGA}\par}
	\vspace{1cm}
	%{\scshape\large \textsc{Extremum Estimators(1)}\par}
	%\vspace{1cm}
	{\Large\bfseries \textsc{2: MODÈLE DE RÉGRESSION LINÉAIRE ET ESTIMATION PAR MCO} \par}
    \vspace{1cm}   
    {\Large\bfseries \textsc{Application}\footnote{Dans ces notes on applique certaines des méthodes et 
	résultats théoriques du cours. Bien que celui-ci puisse être brièvement rappelé. 
	pour plus de détails veuillez vous reporter à celui-ci. Remarquons aussi que les notations 
	utilisées peuvent être différentes par rapport au cours.} \par}
	\vspace{1cm}
	{(\textsc{Cette version: \today})\par}
	\vspace{1cm}
	{\large \textsc{Michal Urdanivia}
	\footnote{Contact:  
	\href{mailto:michal.wong-urdanivia@univ-grenoble-alpes.fr}{michal.wong-urdanivia@univ-grenoble-alpes.fr}, 
	 Université de Grenoble Alpes,  Faculté d'\'Economie, GAEL.}\par}
	%\vfill
	%supervised by\par
	%Dr.~Mark \textsc{Brown}
%\vfill
% Bottom of the page
	
\end{titlepage}


\newpage

\tableofcontents

\newpage

\section{Projection et moindres carrés}

On considère un couple  $(y_i, \boldx_i) \in \R\times \R^K$ qui sont des variables aléatoires pour lesquelles 
on observe un échantillon d'observations i.i.d., $\{(y_i, \boldx_i) ; i=1, \ldots, N\}$. Dans tout ce qui suit 
on suppose que $\Exp[\boldx_i \boldx_i^\prime]$ et $\Exp[y_i^2]$ sont finis.

Définissons le paramètre de la projection de $y_i$ sur $\boldx_i$ par,

\begin{align*}
	\boldgamma &:=\arg\min_{\boldg}\Exp\left[(y_i - \boldx_i^\prime \boldg)^2\right]
\end{align*}

où $\boldgamma$ vérifie les conditions du premier ordre:

\begin{align*}
\Exp\left[\boldx_i(y_i - \boldx_i^\prime\boldgamma)\right] &=0,
\end{align*}
et dès lors que $\Exp[\boldx_i\boldx_i^\prime]$ est de rang plein ce qui revient 
à l'absence de multicolinéarité des éléments de $\boldx_i$, $\boldgamma$ est donnée par l'expression: 
\begin{align*}
	\boldgamma &= \Exp[\boldx_i\boldx_i^\prime]^{-1}\Exp[\boldx_iy_i].
\end{align*}
On peut définir le résidu de la projection,
\begin{align*} 
e_i &= y_i - \boldx_i^\prime\boldgamma,
\end{align*}
et obtenir la décomposition de $y_i$ suivante:

\begin{align*} 
y_i &= \boldx_i^\prime\boldgamma + e_i, \quad \Exp[\boldx_i e_i] = 0.
\end{align*}
On remarque que cette décomposition ne nécessite aucune hypothèse de linéarité.

D'une façon similaire on peut définir le paramètre de la projection de $y_i$ sur sur $\boldx_i$ dans
l'échantillon ou estimateur des moindres carrés par,

\begin{align*}
	\hat{\boldgamma} &:=\arg\min_{\boldg}\Expbar\left[(y_i - \boldx_i^\prime \boldg)^2\right].
\end{align*}
où $\Expbar[f(w_i)] = \frac{1}{N}\sum_{i=1}^N f(w_i)$ est l'abbreviation de la moyenne empirique de 
$f(w_i)$ pour les différentes valeurs de $w_i$ dans l'échantillon de $N$ observations.

$\hat{\boldgamma}$ vérifie les conditions du premier ordre:


\begin{align*}
	\Expbar\left[\boldx_i(y_i - \boldx_i^\prime\hat{\boldgamma})\right] &=0,
	\end{align*}
	et dès lors que $\Expbar[\boldx_i\boldx_i^\prime]$ est de rang plein ce qui revient 
	à l'absence de multicolinéarité des éléments de $\boldx_i$ dans l'échantillon, $\boldgamma$ est donnée par l'expression: 
	\begin{align*}
		\hat{\boldgamma} &= \Expbar[\boldx_i \boldx_i^\prime]^{-1}\Expbar[\boldx_iy_i].
	\end{align*}
	On peut définir,
	\begin{align*} 
	\hat{e}_i &= y_i - \boldx_i^\prime\hat{\boldgamma},
	\end{align*}
	et obtenir la décomposition de $y_i$ suivante:
	
	\begin{align*} 
	y_i &= \boldx_i^\prime\hat{\boldgamma} + \hat{e}_i, \quad \Expbar[\boldx_i \hat{e}_i] = 0.
\end{align*}

\section{Théorème de Frisch-Waugh-Lovell}

Considérons une partition de $\boldx_i$ en deux groupes $\boldx_{1, i}$ et $\boldx_{2,i}$ de respectivement $K_1$ et $K_2$ 
éléments(on suppose que le régresseur constant appartient à  $\boldx_{2i}$) et écrivons:

\begin{align}
y_i &= \boldx_{1, i}^\prime \boldgamma_1 + \boldx_{1, i}^\prime \boldgamma_2 + e_i
\label{eq1}
\end{align}

où l'on a décomposé le paramètre de la projection de $y_i$ sur $\boldx_i$ 
conformément à la décomposition de $\boldx_i$.

Définissons l'opérateur de Frisch-Waugh-Lovell par rapport à un vecteur de variables aléatoires $\boldw_i$ telle que $\Exp[\boldw_i\boldw_i^\prime]
$ soit fini, celui qui appliqué à une variable aléatoire $v_i$ telle que $\Exp[v_i^2]$ soit finie, lui associe le résidu de sa projection sur $\boldw_i$ 
que l'on note $\tilde{v}_i$ avec donc:

\begin{align*}
\tilde{v}_i &= v_i -\boldw_i^\prime\boldgamma_{\boldw_i}, \quad
\boldgamma_{\boldw_i} = \arg\min_{\boldg}\Exp\left[(v_i - \boldw_i^\prime \boldg)^2\right]
\end{align*}
Quand nous appliquons cette opérateur à un vecteur $\boldv_i$ on notera $\tilde{\boldv}_i$ 
le vecteur qui empile les résultats de cette application sur chaque élément de  $\boldv_i$.

Il est facile de vérifier que cet opérateur est linéaire au sens où pour $y_i = v_i + u_i$ avec
$\Exp[v_i^2] + \Exp[u_i^2]<\infty$:

\begin{align*}
\tilde{y}_i &= \tilde{v}_i + \tilde{u}_i.
\end{align*}

Quand nous appliquons cette opérateur sur les termes de \eqref{eq1} nous obtenons,

\begin{align*}
\tilde{y}_i &= \tilde{\boldx}_{1, i}^\prime\boldgamma_1 + \tilde{\boldx}_{2, i}^\prime\boldgamma_2 +
\tilde{e}_i,
\end{align*}
ce qui implique que: 
\begin{align}
	\tilde{y}_i &= \tilde{\boldx}_{1, i}^\prime\boldgamma_1 + e_i, \quad \Exp[ \tilde{\boldx}_{1, i} e_i]=0.
	\label{eq2}
\end{align}
La dernière ligne résulte de ce que $\tilde{\boldx}_{2, i} = 0$ par définition, et de ce que $\tilde{e}_i = e_i$ 
en raison de l'orthogonalité de $e_i$ et $\boldx_i$, $\Exp[\boldx_ie_i] = 0$; et de plus comme 
$\tilde{\boldx}_{1, i}$ 
est une combinaison linéaire d'éléments de $\boldx_i$ nous avons que $\Exp[\tilde{\boldx}_{1, i}e_i]=0$.

La condition $\Exp[ \tilde{\boldx}_{1, i} e_i]=0$ est une condition du premier ordre dans la projection de 
$\tilde{y}_i$ sur $\tilde{\boldx}_{1, i}$. Autrement dit le paramètre $\boldgamma_1$ 
peut être obtenu par projection de $\tilde{y}_i$ sur $\tilde{\boldx}_{1, i}$:

\begin{align*}
\boldgamma_1 &:= \arg\min_{\boldg}\Exp\left[(\tilde{y}_i - \tilde{\boldx}_{1, i}^\prime \boldg)^2\right] = 
\Exp[\tilde{\boldx}_{1, i}\tilde{\boldx}_{1, i}^\prime]^{-1}\Exp[\tilde{\boldx}_{1, i}\tilde{y}_i]
\end{align*}

De manière similaire on peut définir l'opérateur de FVL dans l'échantillon par:

\begin{align*}
	\check{v}_i &= v_i -\boldw_i^\prime\hat{\boldgamma}_{\boldw_i}, \quad
	\hat{\boldgamma}_{\boldw_i} = \arg\min_{\boldg}\Expbar\left[(v_i - \boldw_i^\prime \boldg)^2\right],
	\end{align*}

	et si nous l'appliquons à:

	\begin{align*}
y_i &= \boldx_{1, i}^\prime \hat{\boldgamma}_1 + \boldx_{1, i}^\prime \hat{\boldgamma}_2 + \hat{e}_i
	\end{align*}

	on obtient:
	\begin{align}
		\check{y}_i &= \check{\boldx}_{1, i}^\prime\hat{\boldgamma}_1 + \hat{e}_i, \quad \Exp[ \check{\boldx}_{1, i} \hat{e}_i]=0.
		\label{eq3}
	\end{align}
ce qui implique,

\begin{align*}
	\hat{\boldgamma}_1 &:= \arg\min_{\boldg}\Expbar\left[(\check{y}_i - \check{\boldx}_{1, i}^\prime \boldg)^2\right] = 
	\Expbar[\check{\boldx}_{1, i}\check{\boldx}_{1, i}^\prime]^{-1}\Expbar[\check{\boldx}_{1, i}\check{y}_i]
\end{align*}

\newpage

\section{Questions}

\begin{enumerate}
	\item Montrez que l'estimateur $\hat{\boldgamma}_1$ obtenu par application du résultat de FVL est convergent pour
	$\hat{\boldgamma}_1$. 
	\item Montrer que sa distribution asymptotique est une loi normale. 
\end{enumerate}

\section{Application empirique}

\bibliographystyle{jpe}
\bibliography{../../Biblio.bib}
 \end{document}
