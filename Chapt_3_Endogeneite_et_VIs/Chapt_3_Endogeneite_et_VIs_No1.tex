%\documentclass[ignorenonframetext, compress, 9pt, xcolor=svgnames]{beamer} 
\documentclass[notes, ignorenonframetext, compress, 11pt, xcolor=svgnames, aspectratio=169]{beamer} 
\usepackage{pgfpages}
\usepackage{pdfpages}
% These slides also contain speaker notes. You can print just the slides,
% just the notes, or both, depending on the setting below. Comment out the want
% you want.
\setbeameroption{hide notes} % Only slide
%\setbeameroption{show only notes} % Only notes
%\setbeameroption{show notes on second screen=right} % Both
\usepackage{amsmath}
\usepackage{amsfonts}
\usepackage{amssymb}
\setbeamercolor{frametitle}{fg=MidnightBlue}

\setbeamercolor{sectionpage title}{bg=MidnightBlue}
\setbeamertemplate{frametitle}[default][center]
%\setbeamertemplate{frametitle}{\color{MidnightBlue}\centering\bfseries\insertframetitle\par\vskip-6pt}
\setbeamerfont{frametitle}{series=\bfseries}
\setbeamerfont{title}{series=\bfseries}
\setbeamerfont{sectionpage}{series=\bfseries}
%\setbeamercolor{section in head/foot}{bg=MidnightBlueBlue}
%\setbeamercolor{author in head/foot}{bg=DarkBlue}
\setbeamercolor{author in head/foot}{fg=MidnightBlue}
%\setbeamercolor{title in head/foot}{bg=White}
\setbeamercolor{title in head/foot}{fg=MidnightBlue}
\setbeamercolor{title}{fg=MidnightBlue}
%\setbeamercolor{date in head/foot}{fg=Brown}
%\setbeamercolor{alerted text}{fg=DarkBlue}
%\usecolortheme[named=DarkBlue]{structure} 
%\usepackage{bbm}
%\usepackage{bbold}
\usepackage{eurosym}
\usepackage{graphicx}
%\usepackage{epstopdf}
\usepackage{hyperref}
\hypersetup{
  colorlinks   = true, %Colours links instead of ugly boxes
  urlcolor     = gray, %Colour for external hyperlinks
  linkcolor    = MidnightBlue, %Colour of internal links
  citecolor   = DarkRed %Colour of citations
}
\usepackage{multirow}
\usepackage{xspace}
\usepackage{listings}
\usepackage{natbib}
%\usepackage[sort&compress,comma,super]{natbib}
\def\newblock{} % To avoid a compilation error about a function \newblock undefined
\usepackage{bibentry}
\usepackage{booktabs}
\usepackage{dcolumn}
\usepackage[greek,frenchb]{babel}
\usepackage[babel=true,kerning=true]{microtype}
\usepackage[utf8]{inputenc}
\usepackage[T1]{fontenc}
\usepackage{natbib}
\renewcommand{\cite}{\citet}
\usepackage{longtable}
\usepackage{eso-pic}

\usepackage{xcolor}
 \colorlet{linkequation}{DarkRed} 
 \newcommand*{\SavedEqref}{}
 \let\SavedEqref\eqref 
\renewcommand*{\eqref}[1]{%
\begingroup \hypersetup{
      linkcolor=linkequation,
linkbordercolor=linkequation, }%
\SavedEqref{#1}%
 \endgroup
}

\newcommand*{\refeq}[1]{%
 \begingroup
\hypersetup{ 
linkcolor=linkequation, 
linkbordercolor=linkequation,
}%
\ref{#1}%
 \endgroup
}

\setbeamertemplate{caption}[numbered]
\setbeamertemplate{theorem}[ams style]
\setbeamertemplate{theorems}[numbered]
%\usefonttheme{serif}
%\usecolortheme{beaver}
%\usetheme{Hannover}
%\usetheme{CambridgeUS}
%\usetheme{Madrid}
%\usecolortheme{whale}
%\usetheme{Warsaw}
%\usetheme{Luebeck}
%\usetheme{Montpellier}
%\usetheme{Berlin}
%\setbeamercolor{titlelike}{parent=structure}
%\setbeamertemplate{headline}[default]
%\setbeamertemplate{footline}[default]
%\setbeamertemplate{footline}[Malmoe]
%\setbeamercovered{transparent}
%\setbeamercovered{invisible}
%\usecolortheme{crane}
%\usecolortheme{dolphin}
%\usepackage{pxfonts}
%\usepackage{isomath}
%\usepackage{mathpazo}
%\usepackage{arev} %     (Arev/Vera Sans)
%\usepackage{eulervm} %_   (Euler Math)
%\usepackage{fixmath} %  (Computer Modern)
%\usepackage{hvmath} %_   (HV-Math/Helvetica)
%\usepackage{tmmath} %_   (TM-Math/Times)
%\usepackage{tgheros}
%\usepackage{cmbright}
%\usepackage{ccfonts} \usepackage[T1]{fontenc}
%\usepackage[garamond]{mathdesign}

%\usepackage{color}
%\usepackage{ulem}

%\usepackage[math]{kurier}
%\usepackage[no-math]{fontspec}
%\setmainfont{Fontin Sans}
%\setsansfont{Fontin Sans}
%\setbeamerfont{frametitle}{size=\LARGE,series=\bfseries}
%%%add 19022021
\usepackage{enumerate}    
\usepackage{dcolumn}
\usepackage{verbatim}
\newcolumntype{d}[0]{D{.}{.}{5}}
%\setbeamertemplate{note page}{\pagecolor{yellow!5}\insertnote}
%\usetikzlibrary{positioning}
%\usetikzlibrary{snakes}
%\usetikzlibrary{calc}
%\usetikzlibrary{arrows}
%\usetikzlibrary{decorations.markings}
%\usetikzlibrary{shapes.misc}
%\usetikzlibrary{matrix,shapes,arrows,fit,tikzmark}
%%%
% suppress navigation bar
\beamertemplatenavigationsymbolsempty
%\usetheme{bunsenMod}
%\setbeamercovered{transparent}
%\setbeamertemplate{items}[circle]
%\usecolortheme[named=CadetBlue]{structure}
%\usecolortheme[RGB={225,64,5}]{structure}
%\definecolor{burntRed}{RGB}{225,64,5}
%\setbeamercolor{alerted text}{fg=burntRed} 
%\usecolortheme[RGB={0,40,110}]{structure}
%\hypersetup{linkcolor=burntRed}
%\hypersetup{urlcolor=burntRed}
%\hypersetup{filecolor=burntRed}
%\hypersetup{citecolor=burntRed}

%\usetheme{bunsenMod}
%\setbeamercovered{transparent}
%\setbeamertemplate{items}[circle]
%\usecolortheme[named=CadetBlue]{structure}
%\usecolortheme[RGB={225,64,5}]{structure}
%\definecolor{burntRed}{RGB}{225,64,5}
%\setbeamercolor{alerted text}{fg=burntRed} 
%\usecolortheme[RGB={0,40,110}]{structure}
%\hypersetup{linkcolor=burntRed}
%\hypersetup{urlcolor=burntRed}
%\hypersetup{filecolor=burntRed}
%\hypersetup{citecolor=burntRed}

%\AtBeginSection[] % Do nothing for \section*
%{ \frame{\sectionpage} }
%\setbeamertemplate{frametitle continuation}{}
\newtheorem{lemme}{Lemme}[section]
%\newtheorem{remarque}{Remarque}
\newcommand{\argmax}{\operatornamewithlimits{arg\,max}}
\newcommand{\argmin}{\operatornamewithlimits{arg\,min}}
\def\inprobLOW{\rightarrow_p}
\def\inprobHIGH{\,{\buildrel p \over \rightarrow}\,} 
\def\inprob{\,{\inprobHIGH}\,} 
\def\indist{\,{\buildrel d \over \rightarrow}\,} 
\def\sima{\,{\buildrel a \over \sim}\,} 
\def\F{\mathbb{F}}
\def\R{\mathbb{R}}
\def\N{\mathbb{N}}
\newcommand{\gmatrix}[1]{\begin{pmatrix} {#1}_{11} & \cdots &
    {#1}_{1n} \\ \vdots & \ddots & \vdots \\ {#1}_{m1} & \cdots &
    {#1}_{mn} \end{pmatrix}}
\newcommand{\iprod}[2]{\left\langle {#1} , {#2} \right\rangle}
\newcommand{\norm}[1]{\left\Vert {#1} \right\Vert}
\newcommand{\abs}[1]{\left\vert {#1} \right\vert}
\renewcommand{\det}{\mathrm{det}}
\newcommand{\rank}{\mathrm{rank}}
\newcommand{\spn}{\mathrm{span}}
\newcommand{\row}{\mathrm{Row}}
\newcommand{\col}{\mathrm{Col}}
\renewcommand{\dim}{\mathrm{dim}}
\newcommand{\prefeq}{\succeq}
\newcommand{\pref}{\succ}
\newcommand{\seq}[1]{\{{#1}_n \}_{n=1}^\infty }
\renewcommand{\to}{{\rightarrow}}
\renewcommand{\L}{{\mathcal{L}}}
\newcommand{\Er}{\mathrm{E}}
\renewcommand{\Pr}{\mathrm{P}}
%\newcommand{\Var}{\mathrm{Var}}
%\newcommand{\Cov}{\mathrm{Cov}}
%\newcommand{\corr}{\mathrm{Corr}}
%\newcommand{\Var}{\mathrm{Var}}
\newcommand{\bias}{\mathrm{Bias}}
\newcommand{\mse}{\mathrm{MSE}}
\providecommand{\Pred}{\mathcal{P}}
\providecommand{\plim}{\operatornamewithlimits{plim}}
\providecommand{\avg}{\frac{1}{n} \underset{i=1}{\overset{n}{\sum}}}
\providecommand{\sumin}{{\sum_{i=1}^n}}
\providecommand{\sumin}{{\sum_{i=1}^n}}
\providecommand{\sumiN}{{\sum_{i=1}^N}}
\providecommand{\sumtT}{{\sum_{t=1}^T}}
\providecommand{\limp}{\overset{p}{\rightarrow}}
\providecommand{\liml}{\overset{L}{\rightarrow}}
%\providecommand{\limp}{\underset{n \rightarrow \infty}{\overset{p}{\longrightarrow}}}
%\providecommand{\limp}{\underset{n \rightarrow \infty}{\overset{p}{\longrightarrow}}}
%\providecommand{\limp}{\overset{p}{\longrightarrow}}
%\providecommand{\limd}{\underset{n \rightarrow \infty}{\overset{d}{\longrightarrow}}}
\providecommand{\limd}{\overset{d}{\rightarrow}}
\providecommand{\limps}{\overset{p.s.}{\rightarrow}}
\providecommand{\limlp}{\overset{L^p}{\rightarrow}}
\def\independenT#1#2{\mathrel{\setbox0\hbox{$#1#2$}%
    \copy0\kern-\wd0\mkern4mu\box0}} 
\newcommand\indep{\protect\mathpalette{\protect\independenT}{\perp}}


\lstset{language=R}
\lstset{keywordstyle=\color[rgb]{0,0,1},                                        % keywords
        commentstyle=\color[rgb]{0.133,0.545,0.133},    % comments
        stringstyle=\color[rgb]{0.627,0.126,0.941}      % strings
}       
\lstset{
  showstringspaces=false,       % not emphasize spaces in strings 
  columns=fixed,
  numbersep=3mm, numbers=left, numberstyle=\tiny,       % number style
  frame=none,
  framexleftmargin=5mm, xleftmargin=5mm         % tweak margins
}
\makeatletter
%\setbeamertemplate{frametitle continuation}{\gdef\beamer@frametitle{}}
\setbeamertemplate{frametitle continuation}{\frametitle{}}
%\setbeamertemplate{frametitle continuation}{\insertcontinuationcount}
\makeatother

\theoremstyle{remark}
\newtheorem{interpretation}{Interprétation}
\newtheorem*{interpretation*}{Interprétation}

\theoremstyle{remark}
\newtheorem{remarque}{Remarque}%[section]
\newtheorem*{remarque*}{Remarque}
\usepackage[framemethod=TikZ]{mdframed} 
\usepackage{showexpl}
%\newtheorem{step}{Step}[section]
%\newtheorem{rem}{Comment}[section]
%\newtheorem{ex}{Example}[section]
%\newtheorem{hist}{History}[section]
%\newtheorem*{ex*}{Example}
\theoremstyle{plain}
\newtheorem{propriete}{Propri\'et\'e}
\renewcommand{\thepropriete}{P\arabic{propriete}}
%\theoremstyle{definition}
%\newtheorem{definition}{Définition}%[section]
\theoremstyle{remark}
\newtheorem{exemple}{Exemple}
\newtheorem*{exemple*}{Exemple}

\newtheorem{theoreme}{Théorème}
\newtheorem{proposition}{Proposition}
%\newtheorem{propriete}{Propri\'et\'e}
\newtheorem{corollaire}{Corollaire}
%\newtheorem{exemple}{Exemple}
\newtheorem{assumption}{Assumption}
\renewcommand{\theassumption}{A\arabic{assumption}}
\newtheorem{hypothese}{Hypothèse}
\renewcommand{\thehypothese}{H\arabic{hypothese}}
\theoremstyle{definition}

%\newtheorem{definitionx}{D\'efinition}%[section]
%\newenvironment{definition}
 %{\pushQED{\qed}\renewcommand{\qedsymbol}{$\triangle$}\definitionx}
 %{\popQED\enddefinitionx}

\newtheorem{condition}{Condition}
\renewcommand{\thecondition}{C\arabic{condition}}
%\newcommand{\Var}{\mathbb{V}}
%\newcommand{\Var}{\mathbf{Var}}
%\newcommand{\Exp}{\mathbf{E}}
%\providecommand{\Vr}{\mathrm{Var}}
%\renewcommand{\Er}{\mathbb{E}}
%\newcommand{\LP}{\mathcal{LP}}
%\providecommand{\Id}{\mathbf{I}}
%\providecommand{\Rang}{\mathrm{Rang}}
%\providecommand{\Trace}{\mathrm{Trace}}
%\newcommand{\Cov}{\mathbf{Cov}}
%\newcommand{\Cov}{\mathbb{C}\mathrm{ov}}
\providecommand{\Id}{\mathbf{I}}
\providecommand{\Ind}{\mathbf{1}}
\providecommand{\uvec}{\mathbf{1}}
\providecommand{\vecOnes}{\mathbf{1}}
\DeclareMathOperator{\indfun}{\mathbf{1}}
\DeclareMathOperator{\Exp}{E}
\DeclareMathOperator{\Expn}{\mathbb{E}_n}
\DeclareMathOperator{\Var}{Var}
\DeclareMathOperator{\Vr}{V}
\DeclareMathOperator{\Cov}{Cov}
\DeclareMathOperator{\corr}{corr}
\DeclareMathOperator{\perps}{\perp_s}
%\DeclareMathOperator{\Prob}{Pr}
\DeclareMathOperator{\Prob}{P}
\DeclareMathOperator{\prob}{p}
\DeclareMathOperator{\loss}{L}
\providecommand{\Corr}{\mathrm{Corr}}
\providecommand{\Diag}{\mathrm{Diag}}
\providecommand{\reg}{\mathrm{r}}
\providecommand{\Likelihood}{\mathrm{L}}
\renewcommand{\Pr}{{\mathbb{P}}}
\providecommand{\set}[1]{\left\{#1\right\}}
\providecommand{\uvec}{\mathbf{1}}
\providecommand{\Rang}{\mathrm{Rang}}
\providecommand{\Trace}{\mathrm{Trace}}
\providecommand{\Tr}{\mathrm{Tr}}
\providecommand{\CI}{\mathrm{CI}}
\providecommand{\asyvar}{\mathrm{AsyVar}}
\DeclareMathOperator{\Supp}{Supp}
\newcommand{\inputslide}[2]{{
    \usebackgroundtemplate{
     \includegraphics[page={#2},width=0.90\textwidth,keepaspectratio=true]
      %\includegraphics[page={#2},width=\paperwidth,keepaspectratio=true]
      {{#1}}}
    \frame[plain]{}
  }}
\newcommand\pperp{\perp\!\!\!\perp}
\newcommand\independent{\protect\mathpalette{\protect\independenT}{\perp}}
\def\independenT#1#2{\mathrel{\rlap{$#1#2$}\mkern2mu{#1#2}}}
\usepackage{bbm}
\providecommand{\Ind}{\mathbf{1}}
\newcommand{\sumjsi}{\underset{i<j}{{\sum}}}
\newcommand{\prodjsi}{\underset{i<j}{{\prod}}}
\newcommand{\sumisj}{\underset{j<i}{{\sum}}}
\newcommand{\prodisj}{\underset{j<i}{{\prod}}}
\newcommand{\sumobs}{\underset{i=1}{\overset{n}{\sum}}}
\newcommand{\sumi}{\underset{i=1}{\overset{n}{\sum}}}
\newcommand{\prodi}{\underset{i=1}{\overset{n}{\prod}}}
\newcommand{\prodobs}{\underset{i=1}{\overset{n}{\prod}}}
\newcommand{\simiid}{{\overset{i.i.d.}{\sim}}}
%\newcommand{\sumobs}{\sum_{i=1}^N}
%\newcommand{\prodobs}{\prod_{i=1}^N}
%\newcommand{\sumjsi}{\sum_{i<j}}
%\newcommand{\prodjsi}{\prod_{i<j}}
%\newcommand{\sumisj}{\sum_{j<i}}
%\newcommand{\prodisj}{\sum_{j<i}}

%\usepackage{appendixnumberbeamer}
\setbeamertemplate{footline}[frame number]
\setbeamertemplate{section in toc}[sections numbered]
\setbeamertemplate{subsection in toc}[subsections numbered]
\setbeamertemplate{subsubsection in toc}[subsubsections numbered]

%\makeatother
%\setbeamertemplate{footline}
%{
%    \leavevmode%
%    \hbox{%
%        \begin{beamercolorbox}[wd=.333333\paperwidth,ht=2.25ex,dp=1ex,center]{author in head/foot}%
%            \usebeamerfont{author in head/foot}\insertshortauthor
%        \end{beamercolorbox}%
%        \begin{beamercolorbox}[wd=.333333\paperwidth,ht=2.25ex,dp=1ex,center]{title in head/foot}%
%            \usebeamerfont{title in head/foot}\insertshorttitle
%        \end{beamercolorbox}%
%        \begin{beamercolorbox}[wd=.333333\paperwidth,ht=2.25ex,dp=1ex,right]{date in head/foot}%
%            \usebeamerfont{date in head/foot}\insertshortdate{}\hspace*{2em}
%            \insertframenumber{} / \inserttotalframenumber\hspace*{2ex} 
%        \end{beamercolorbox}}%
%       \vskip0pt%
 %   }
%   \makeatother
%\setbeamertemplate{navigation symbols}{}
\setbeamertemplate{itemize items}[ball]
%\setbeamertemplate{itemize items}{-}
%\newenvironment{wideitemize}{\itemize\addtolength{\itemsep}{10pt}}{\enditemize}
% \usepackage{eso-pic}
%\newcommand\AtPagemyUpperLeft[1]{\AtPageLowerLeft{%
%\put(\LenToUnit{0.9\paperwidth},\LenToUnit{0.9\paperheight}){#1}}}
%\AddToShipoutPictureFG{
%  \AtPagemyUpperLeft{{\includegraphics[width=1.1cm,keepaspectratio]{../logo-uga.png}}}
%}%
\def\figheight{3in}
\def\figwidth{4in}

%%Commands from Econometric Theory(Slides) by J. Stachurski.

\newcommand{\boldx}{ {\mathbf x} }
\newcommand{\boldu}{ {\mathbf u} }
\newcommand{\boldv}{ {\mathbf v} }
\newcommand{\boldw}{ {\mathbf w} }
\newcommand{\boldy}{ {\mathbf y} }
\newcommand{\boldb}{ {\mathbf b} }
\newcommand{\bolda}{ {\mathbf a} }
\newcommand{\boldc}{ {\mathbf c} }
\newcommand{\boldd}{ {\mathbf d} }
\newcommand{\boldi}{ {\mathbf i} }
\newcommand{\bolde}{ {\mathbf e} }
\newcommand{\boldp}{ {\mathbf p} }
\newcommand{\boldq}{ {\mathbf q} }
\newcommand{\bolds}{ {\mathbf s} }
\newcommand{\boldt}{ {\mathbf t} }
\newcommand{\boldz}{ {\mathbf z} }
\newcommand{\boldr}{ {\mathbf r} }

\newcommand{\boldzero}{ {\mathbf 0} }
\newcommand{\boldone}{ {\mathbf 1} }

\newcommand{\boldalpha}{ {\boldsymbol \alpha} }
\newcommand{\boldbeta}{ {\boldsymbol \beta} }
\newcommand{\boldgamma}{ {\boldsymbol \gamma} }
\newcommand{\boldtheta}{ {\boldsymbol \theta} }
\newcommand{\boldxi}{ {\boldsymbol \xi} }
\newcommand{\boldtau}{ {\boldsymbol \tau} }
\newcommand{\boldepsilon}{ {\boldsymbol \epsilon} }
\newcommand{\boldvepsilon}{ {\boldsymbol \varepsilon} }
\newcommand{\boldmu}{ {\boldsymbol \mu} }
\newcommand{\boldSigma}{ {\boldsymbol \Sigma} }
\newcommand{\boldOmega}{ {\boldsymbol \Omega} }
\newcommand{\boldPhi}{ {\boldsymbol \Phi} }
\newcommand{\boldLambda}{ {\boldsymbol \Lambda} }
\newcommand{\boldphi}{ {\boldsymbol \phi} }

\newcommand{\Sigmax}{ {\boldsymbol \Sigma_{\boldx}}}
\newcommand{\Sigmau}{ {\boldsymbol \Sigma_{\boldu}}}
\newcommand{\Sigmaxinv}{ {\boldsymbol \Sigma_{\boldx}^{-1}}}
\newcommand{\Sigmav}{ {\boldsymbol \Sigma_{\boldv \boldv}}}

\newcommand{\hboldx}{ \hat {\mathbf x} }
\newcommand{\hboldy}{ \hat {\mathbf y} }
\newcommand{\hboldb}{ \hat {\mathbf b} }
\newcommand{\hboldu}{ \hat {\mathbf u} }
\newcommand{\hboldtheta}{ \hat {\boldsymbol \theta} }
\newcommand{\hboldtau}{ \hat {\boldsymbol \tau} }
\newcommand{\hboldmu}{ \hat {\boldsymbol \mu} }
\newcommand{\hboldbeta}{ \hat {\boldsymbol \beta} }
\newcommand{\hboldgamma}{ \hat {\boldsymbol \gamma} }
\newcommand{\hboldSigma}{ \hat {\boldsymbol \Sigma} }

\newcommand{\boldA}{\mathbf A}
\newcommand{\boldB}{\mathbf B}
\newcommand{\boldC}{\mathbf C}
\newcommand{\boldD}{\mathbf D}
\newcommand{\boldI}{\mathbf I}
\newcommand{\boldL}{\mathbf L}
\newcommand{\boldM}{\mathbf M}
\newcommand{\boldP}{\mathbf P}
\newcommand{\boldQ}{\mathbf Q}
\newcommand{\boldR}{\mathbf R}
\newcommand{\boldX}{\mathbf X}
\newcommand{\boldU}{\mathbf U}
\newcommand{\boldV}{\mathbf V}
\newcommand{\boldW}{\mathbf W}
\newcommand{\boldY}{\mathbf Y}
\newcommand{\boldZ}{\mathbf Z}

\newcommand{\bSigmaX}{ {\boldsymbol \Sigma_{\hboldbeta}} }
\newcommand{\hbSigmaX}{ \mathbf{\hat \Sigma_{\hboldbeta}} }
\newcommand{\betahat}{\hat{\beta}}
\newcommand{\gammahat}{\hat{\gamma}}
\newcommand{\Uhat}{\hat{U}}
\newcommand{\Vhat}{\hat{V}}
\newcommand{\epsilonhat}{\hat{\epsilon}}
\newcommand{\sigmahat}{\hat{\sigma}}
\newcommand{\Sigmahat}{\hat{\Sigma}}
\newcommand{\Gammahat}{\hat{\Gamma}}

\newcommand{\RR}{\mathbbm R}
\newcommand{\CC}{\mathbbm C}
\newcommand{\NN}{\mathbbm N}
\newcommand{\PP}{\mathbbm P}
\newcommand{\EE}{\mathbbm E \nobreak\hspace{.1em}}
\newcommand{\EEP}{\mathbbm E_P \nobreak\hspace{.1em}}
\newcommand{\ZZ}{\mathbbm Z}
\newcommand{\QQ}{\mathbbm Q}


\newcommand{\XX}{\mathcal X}

\newcommand{\aA}{\mathcal A}
\newcommand{\fF}{\mathscr F}
\newcommand{\bB}{\mathscr B}
\newcommand{\iI}{\mathscr I}
\newcommand{\rR}{\mathscr R}
\newcommand{\dD}{\mathcal D}
\newcommand{\lL}{\mathcal L}
\newcommand{\llL}{\mathcal{H}_{\ell}}
\newcommand{\gG}{\mathcal G}
\newcommand{\hH}{\mathcal H}
\newcommand{\nN}{\textrm{\sc n}}
\newcommand{\lN}{\textrm{\sc ln}}
\newcommand{\pP}{\mathscr P}
\newcommand{\qQ}{\mathscr Q}
\newcommand{\xX}{\mathcal X}

\newcommand{\ddD}{\mathscr D}


%\newcommand{\R}{{\texttt R}}
\newcommand{\risk}{\mathcal R}
\newcommand{\Remp}{R_{{\rm emp}}}

\newcommand*\diff{\mathop{}\!\mathrm{d}}
\newcommand{\ess}{ \textrm{{\sc ess}} }
\newcommand{\tss}{ \textrm{{\sc tss}} }
\newcommand{\rss}{ \textrm{{\sc rss}} }
\newcommand{\rssr}{ \textrm{{\sc rssr}} }
\newcommand{\ussr}{ \textrm{{\sc ussr}} }
\newcommand{\zdata}{\mathbf{z}_{\mathcal D}}
\newcommand{\Pdata}{P_{\mathcal D}}
\newcommand{\Pdatatheta}{P^{\mathcal D}_{\theta}}
\newcommand{\Zdata}{Z_{\mathcal D}}


\newcommand{\e}[1]{\mathbbm{E}[{#1}]}
\newcommand{\p}[1]{\mathbbm{P}({#1})}
% definition
\BeforeBeginEnvironment{definition}{
  \setbeamerfont{block title}{series=\bfseries}
  \setbeamercolor{block title}{fg=MidnightBlue,bg=white}
  \setbeamercolor{block body}{fg=black, bg=gray!10}
}
\newtheorem*{definition*}{Definition}
\BeforeBeginEnvironment{definition*}{
  \setbeamerfont{block title}{series=\bfseries}
  \setbeamercolor{block title}{fg=MidnightBlue,bg=white}
  \setbeamercolor{block body}{fg=black, bg=gray!10}
}

% theorem
\BeforeBeginEnvironment{theorem}{
  \setbeamerfont{block body}{shape=\itshape}
  \setbeamerfont{block title}{series=\bfseries}
  \setbeamercolor{block title}{fg=MidnightBlue,bg=white}
  \setbeamercolor{block body}{fg=black, bg=gray!10}
}
\newtheorem*{theorem*}{Theorem}
\BeforeBeginEnvironment{theorem*}{
  \setbeamerfont{block body }{shape=\itshape}
  \setbeamerfont{block title}{series=\bfseries}
  \setbeamercolor{block title}{fg=MidnightBlue,bg=white}
  \setbeamercolor{block body}{fg=black, bg=gray!10}
}

% definition_fr
\theoremstyle{definition}
\newtheorem{definition_fr}{Définition}%[section]
\BeforeBeginEnvironment{definition_fr}{
  \setbeamerfont{block title}{series=\bfseries}
  \setbeamercolor{block title}{fg=MidnightBlue,bg=white}
  \setbeamercolor{block body}{fg=black, bg=gray!10}
}
\newtheorem*{definition_fr*}{Définition}
\BeforeBeginEnvironment{definition_fr*}{
  \setbeamerfont{block title}{series=\bfseries}
  \setbeamercolor{block title}{fg=MidnightBlue,bg=white}
  \setbeamercolor{block body}{fg=black, bg=gray!10}
}
% theorem_fr
\newtheorem{theorem_fr}{Théorème}%[section]
\BeforeBeginEnvironment{theorem_fr}{
  \setbeamerfont{block body}{shape=\itshape}
  \setbeamerfont{block title}{series=\bfseries, shape = \upshape}
  \setbeamercolor{block title}{fg=MidnightBlue,bg=white}
  \setbeamercolor{block body}{fg=black, bg=gray!10}
}
\newtheorem*{theorem_fr*}{Théorème}
\BeforeBeginEnvironment{theorem_fr*}{
  \setbeamerfont{block body}{shape=\itshape}
  \setbeamerfont{block title}{series=\bfseries, shape = \upshape}
  \setbeamercolor{block title}{fg=MidnightBlue,bg=white}
  \setbeamercolor{block body}{fg=black, bg=gray!10}
}

% remark_fr
\theoremstyle{remark}
\newtheorem{remark_fr}{Remarque}%[section]
\BeforeBeginEnvironment{remark_fr}{
  \setbeamerfont{block title}{series=\bfseries, shape=\itshape}
  \setbeamercolor{block title}{fg=MidnightBlue,bg=white}
  \setbeamercolor{block body}{fg=black, bg=gray!10}
}
\newtheorem*{remark_fr*}{Remarque}
\BeforeBeginEnvironment{remark_fr*}{
  \setbeamerfont{block title}{series=\bfseries, shape=\itshape}
  \setbeamercolor{block title}{fg=MidnightBlue,bg=white}
  \setbeamercolor{block body}{fg=black, bg=gray!10}
}








\usepackage{color}
\usepackage{tikz}
\usetikzlibrary{shapes.geometric, arrows}
\usepackage{enumerate}   
\usepackage{multirow}
%\setbeamersize{text margin left=1.5em,text margin right=1.5em} 
%\setbeamersize{text margin left=1.2cm,text margin right=1.2cm} 
\setbeamersize{text margin left=1.5em,text margin right=1.5em} 
%\usepackage{xr}
%\externaldocument{Econometrie1_UGA_P2e}
  \usepackage{eso-pic}
%\newcommand\AtPagemyUpperLeft[1]{\AtPageLowerLeft{%
%\put(\LenToUnit{0.9\paperwidth},\LenToUnit{0.85\paperheight}){#1}}}
%\AddToShipoutPictureFG{
 % \AtPagemyUpperLeft{{\includegraphics[width=1.1cm,keepaspectratio]{logoUGA2020.pdf}}}
%}%

%\setbeamercolor{title}{fg=black}
%\setbeamercolor{frametitle}{fg=black}
%\setbeamercolor{section in head/foot}{fg=black}
%\setbeamercolor{author in head/foot}{bg=Brown}
%\setbeamercolor{date in head/foot}{fg=Brown}
\setbeamertemplate{section page}
{
    \begin{centering}
    \begin{beamercolorbox}[sep=11pt,center]{part title}
    \usebeamerfont{section title}\thesection.~\insertsection\par
    \end{beamercolorbox}
    \end{centering}
}
%\titlegraphic{\includegraphics[width=1cm]{logoUGA2020.pdf}}
\title[Regression linéaire]{\textbf{ \'ECONOM\'ETRIE \\ (UGA, S2)}}
\subtitle{\textbf{CHAPITRE 3:\\ ENDOGÉNÉITÉ ET VARIABLES INSTRUMENTALES(1)}}
\date{\today}
\author{Michal W. Urdanivia\inst{*}}
\institute{\inst{*}UGA, Facult\'e d'\'Economie, GAEL, \\
e-mail:
 \href{
     mailto:michal.wong-urdanivia@univ-grenoble-alpes.fr}{michal.wong-urdanivia@univ-grenoble-alpes.fr}}

%\titlegraphic{\includegraphics[width=1cm]{logoUGA2020.pdf}
%}

\begin{document}

%%% TIKZ STUFF
\usetikzlibrary{positioning}
\usetikzlibrary{snakes}
\usetikzlibrary{calc}
\usetikzlibrary{arrows}
\usetikzlibrary{decorations.markings}
\usetikzlibrary{shapes.misc}
\usetikzlibrary{matrix,shapes,arrows,fit,tikzmark}
\usetikzlibrary{shapes}
\usetikzlibrary{shapes.geometric, arrows}
\tikzset{   
        every picture/.style={remember picture,baseline},
        every node/.style={anchor=base,align=center,outer sep=1.5pt},
        every path/.style={thick},
        }
\newcommand\marktopleft[1]{
    \tikz[overlay,remember picture] 
        \node (marker-#1-a) at (-.3em,.3em) {};%
}
\newcommand\markbottomright[2]{%
    \tikz[overlay,remember picture] 
        \node (marker-#1-b) at (0em,0em) {};%
}
\tikzstyle{every picture}+=[remember picture] 
\tikzstyle{mybox} =[draw=black, very thick, rectangle, inner sep=10pt, inner ysep=20pt]
\tikzstyle{fancytitle} =[draw=black,fill=red, text=white]
\tikzstyle{observed}=[draw,circle,fill=gray!50]

\begin{frame}
\titlepage
\end{frame}
\begin{frame}
 \tableofcontents
    \end{frame}
%\begin{frame}
%\frametitle{Contenu}
%\tableofcontents[pausesections, pausesubsections]
%\end{frame}

%\section{Qu'est-ce que l’économétrie ? A quoi (à qui) ça sert ?}
%\frame{\sectionpage}
%\begin{frame}
%  \tableofcontents  
%\end{frame}

\section{Introduction à la notion de variable instrumentale}
\frame{\sectionpage}
\begin{frame}[allowframebreaks]{Endogénéité et (non-)identification dans un modèle linéaire simple}
\begin{itemize}
\item On considère ici le modèle linéaire le plus simple :
\begin{align}
    y_i &=\alpha_0 +b_0x_i + u_i \  \text{avec} \ \Exp[u_i]:= 0,
    \label{eq1}
\end{align}
mais on considère que l’analyse du PGD indique que $x_i$ est endogène dans ce 
modèle, i.e. que :
\begin{align*}
\Exp[u_i| x_i]\neq 0 &\Rightarrow \Cov[x_i; u_i]\neq 0.
\end{align*}
\item Rappelons que lorsque $x_i$ est exogène et que $\Var[x_i] \neq 0$, on a par l'exogénéité de 
$x_i$:
\begin{align*}
    \Cov[x_i; u_i] = 0 &\Leftrightarrow \Cov[x_i; y_i - \alpha_0 +b_0x_i] =0
     \Rightarrow b_0 = \frac{\Cov[x_i; y_i]}{\Var[x_i]}.
\end{align*}
\item Autrement dit l'exogénéité de $x_i$ permet d'identifier $b_0$(et aussi de $\alpha_0$)
comme une fonction de la distribution des variables observées $y_i$ et $x_i$.
\item Inversement dans la situation que nous considérons dans ce chapitre  $\Cov[x_i; u_i]\neq 0$, 
rend impossible l'identification $b_0$(et aussi celle de $\alpha_0$), et 
la construction d'un estimateur convergent.

\framebreak

\item On peut représenter ce problème en utilisant un
\href{https://en.wikipedia.org/wiki/Causal\_graph}{\textbf{graphe causal}}: 

\begin{figure}[hbt!]
    \centering
    \begin{tikzpicture}
           \node[draw,circle, fill=gray!50](xtilde)  at (0,0) {$x_i$};
          \node[draw,circle,fill=gray!50](y)  at (0,-2) {$y_i$};
      \node[draw,circle](u)  at (2, -2) {$u_i$};
      \draw[->,>=latex, line width= 1] (xtilde) -- (y);
       \draw[->,>=latex, dashed, line width= 1] (u) -- (y);
        \draw[<->,>=latex ,dashed,  line width= 1] (u) -- (xtilde);
          \end{tikzpicture}
          \caption{Graphe causal du modèle: $y_i = \alpha_0 + b_0x_i + u_i$, 
          avec $\Cov(x_i ; u_i)\neq 0$. Les variables
          $(x_i, y_i, u_i)$ sont les nœuds du graph et les 
           nœuds foncés correspondent aux variables observées.
          Les arêtes représentent les relations entre les variables. 
          Les relations observées sont en trait plein.}
    \label{fig1}
          \end{figure}
        \end{itemize}
\end{frame}

\begin{frame}[allowframebreaks]{Identification avec une variable instrumentale}
    \begin{itemize}
\item L'intuition sous-jacente à la méthode des VIs consiste à répondre à la question 
de savoir si avec une variable, que nous notons $z_i$, il est possible d'obtenir 
une mesure de la relation causale entre $x_i$ et $y_i$ qui ne dépende pas de $u_i$.
\item Autrement dit, $z_i$ doit être exogène par rapport à $u_i$:
\begin{align}
\Exp[u_i| z_i] = 0 &\Rightarrow \Cov[z_i; u_i]= 0,
    \label{eq2}
\end{align}
ce qui nous permets d'écrire:
\begin{align*}
    \Cov[z_i; u_i]= 0 &\Leftrightarrow 
    \Cov[z_i; y_i - \alpha_0 +b_0x_i] = 0\\
    &\Leftrightarrow \Cov[z_i; y_i - \alpha_0 +b_0\Cov[z_i;x_i] = 0
\end{align*}
\item Ceci indique que pour identifier $b_0$  on doit aussi supposer aussi que,
\begin{align}
\Cov[z_i;x_i]  &\neq 0,
\label{eq3}
\end{align}
et $b_0$ est identifié par:
\begin{align}
b_0 &= \frac{\Cov[z_i; y_i]}{\Cov[z_i;x_i]},
    \label{eq4}
\end{align}
\framebreak

\item On peut résumer les conditions \eqref{eq2}-\eqref{eq3} ainsi: 

\begin{definition_fr}[Conditions de validité de VIs dans un modèle simple]
    Dans le modèle $y_i = \alpha + b_0x_i + u_i$ avec $\Exp[u_i] := 0$ , une variable $z_i$ 
    est un instrument(de $x_i$) ssi:
    \begin{enumerate}[(i)]
\item $\Cov[z_i ;u_i ]= 0$, i.e. $z_i$ est exogène par rapport à $u_i$ et:
\item $\Cov[z_i;x_i]$, i.e. $z_i$ et $x_i$ sont liées.
\end{enumerate}
\end{definition_fr}

\item Dans la représentation en termes de graphe causal cela donne:

\begin{figure}[hbt!]
    \centering
    \begin{tikzpicture}
           \node[draw,circle, fill=gray!50](xtilde)  at (0,0) {$x_i$};
           \node[draw,circle,fill=gray!50](ztilde)  at (-2, 0) {$z_i$};
          \node[draw,circle,fill=gray!50](y)  at (0,-2) {$y_i$};
      \node[draw,circle](u)  at (2, -2) {$u_i$};
      \draw[->,>=latex, line width= 1] (xtilde) -- (y);
      \draw[<->,>=latex, line width= 1] (ztilde) -- (xtilde);
       \draw[->,>=latex, dashed, line width= 1] (u) -- (y);
        \draw[<->,>=latex ,dashed,  line width= 1] (u) -- (xtilde);
          \end{tikzpicture}
          \caption{Graphe causal du modèle: $y_i = \alpha_0 + b_0x_i + u_i$, 
          avec $\Cov(x_i ; u_i)\neq 0$, $\Cov(z_i ; u_i) = 0$. Les variables
          $(z_i, x_i, y_i, u_i)$ sont les nœuds du graph et les 
           nœuds foncés correspondent aux variables observées.
          Les arêtes représentent les relations entre les variables. 
          Les relations observées sont en trait plein.}
    \label{fig2}
          \end{figure}

\end{itemize}
\end{frame}
\begin{frame}[allowframebreaks]{Estimateur des VIs dans le modèle simple}
    \begin{itemize}
        \item L'identification de $b_0$ par \eqref{eq4} suggère l'estimateur:
        \begin{align}
            \hat{b}_N^{VI} &= \frac{N^{-1}\sumiN(z_i - \bar{z}_N)(y_i - \bar{y}_N)}{
                N^{-1}\sumiN(z_i - \bar{z}_N)(x_i - \bar{x}_N)} = 
                \frac{N^{-1}\sumiN (z_-\bar{z}_N)y_i}
                {N^{-1}\sumiN (z_-\bar{z}_N)x_i},
                \label{eq5}
        \end{align}
        où $\bar{z}_N$, $\bar{x}_N$, et  $\bar{y}_N$ sont le moyennes empiriques respectives de 
        $z_i$, $x_i$, et $y_i$. 
        \item De plus, $\hat{b}_N^{VI}$ est convergent. Nous avons en effet:
        \begin{align*}
            \underset{N\to + \infty}{\plim} N^{-1}\sumiN (z_-\bar{z}_N)y_i &\rightarrow 
            \Cov[z_i; y_i],\\
            \underset{N\to + \infty}{\plim} N^{-1}\sumiN (z_-\bar{z}_N)x_i &\rightarrow 
            \Cov[z_i; x_i],
        \end{align*}
        d'où:
        \begin{align*}
            \hat{b}_N^{VI} \underset{N\to + \infty}{\limp} 
            \frac{\Cov[z_i; y_i]}{\Cov[z_i;x_i]} &= \frac{\Cov[z_i; \alpha_0 + b_0x_i + u_i]}{\Cov[z_i;x_i]},\\
            &= b_0 + \frac{\Cov[z_i;u_i]}{\Cov[z_i;x_i]},\\
            &=b_0.
        \end{align*}
    \end{itemize}
    \framebreak
    \begin{remark_fr}
        \begin{enumerate}[$\star$]
            \item On dit des variations de $z_i$ qu’elles sont des variations exogènes: 
            elles ne sont pas liées à $u_i$ puisque $\Cov[z_i;u_i]=0$.
            \item Ce sont les effets de ces variations exogènes sur $x_i$ qui sont exploitées pour 
            l’identification de $b_0$ grâce à $\Cov[z_i;x_i]\neq 0$.
            \item Noter qu’il n’est aucunement nécessaire que l’effet de $z_i$ sur $x_i$ soit causal. 
            \item L’effet de $z_i$ sur $y_i$ ne « transite » que via $x_i$. 
            La variable instrumentale $z_i$ n’est pas une variable explicative 
            dans le modèle de $y_i$. 
            On parle alors de relation d’exclusion (de la VI $z_i$ vis-à-vis du modèle de $y_i$). 
            \item L'estimateur des VIs est parfois appelé estimateur des moindres carrés indirects. 
            Cela provient de ce que $b_0$ dans $\eqref{eq4}$ peut s'écrire: 
            \begin{align*}
                b_0 &= \frac{\Cov[z_i; y_i] / \Var[z_i]}{\Cov[z_i;x_i] / \Var[z_i]},
                \end{align*}
            qui est le rapport entre le coefficient de $z_i$ dans la projection de $y_i$ sur $z_i$, 
            et le coefficient de $z_i$ dans la projection de $x_i$ sur $z_i$.
        \end{enumerate}
    \end{remark_fr}
    \end{frame}   
\section{L'estimateur de VIs}
\frame{\sectionpage}
\begin{frame}[allowframebreaks]{Variables endogènes, exogènes, instruments}
\begin{itemize}
    \item L’objectif est maintenant de généraliser l’approche présentée dans
     le cas simple précédent au  modèle linéaire général:
    \begin{align}
        y_i&=\boldx_i^\prime\bolda_0 + u_i, \ \text{avec} \ \Exp[u_i] := 0.
        \label{eq6}
    \end{align}
    \item Plusieurs éléments du vecteur $\boldx_i$ 
    peuvent être endogènes de sorte que dans l'estimateur des MCO de $\bolda_0$ plusieurs
    éléments sont potentiellement biaisés(c.f. cours précédent sur les VIs). 
    \item Notons:

    \[\boldx_i = 
    \begin{bmatrix}
        \begin{bmatrix}
        1\\
        \tilde{\boldx}_i^x
        \end{bmatrix}\\
        \tilde{\boldx}_i^e
    \end{bmatrix} 
    =
    \begin{bmatrix}
        \boldx_i^x\\
        \tilde{\boldx}_i^e
    \end{bmatrix}
    \begin{array}{ll}
        \left\{ \text{variables explicatives exogènes}\right. &: \Exp[u_i|x_{k, i}^x] = 0 (k=1,\ldots,M)\\
        \left\{ \text{variables explicatives endogènes}\right.&: \Exp[u_i|x_{k, i}^x] \neq 0(k=M +1,\ldots,K)
    \end{array}
    \]
    \begin{remark_fr}
    \begin{enumerate}[$\star$]
        \item Il est clair que la variable constante $1$ est «exogène» :$\Exp[1\times u_i] =\Exp[ui|] 
        =\Exp[ u_i] =0$.
        \item Comme pour l'estimateur des MCO nous utiliserons la Méthode des Moments 
        pour construire un estimateur convergent de $\bolda_0$ , l’estimateur des VI 
        du modèle \eqref{eq6}.
        \item On considère ici que chaque élément $\boldx_i^e$ a une variable instrumentale.
    \end{enumerate}
    \end{remark_fr}

    \framebreak
    \begin{definition_fr}[Variable instrumentale]
        $z_{k ,i}$  est une variable instrumentale de $x_{k, i}$ dans le modèle linéaire
         \eqref{eq6} si: 
         \begin{enumerate}[(i)]
            \item $\Cov[z_{k, i}; u_i] = 0$ i.e., $z_{k, i}$ est exogènes par rapport à $u_i$,
            \label{vi1}
            \item $z_{k ,i}$ « suffisamment » liée à $x_{k ,i}$.
            \label{vi2}
         \end{enumerate}
    \end{definition_fr}
    \begin{remark_fr}
        \begin{enumerate}[$\star$]
        \item On verra dans la suite (analyse des conditions de rang) que la condition 
        \eqref{vi2} doit en fait être définie comme:
        \begin{align*}
            \Cov[z_{k, i} ; e_{k, i} ]&\neq 0 \  \text{pour} \ k > 1,
        \end{align*}
        où $e_{k, i}$ est la partie spécifique de $x_{k ,i}$ dans $x_i$ , i.e. 
        le résidu de la projection linéaire de $x_{k ,i}$ sur les autres explicatives $\boldx_{-k, i}$.
        \begin{align*}
            e_{k, i} &= x_{k, i}- \mathcal{EL}[x_{k, i}| \boldx_{-k, i}].
      \end{align*}
      \item Dans la définition d'un VI précédente, on voit que lorsqu'une variable explicative $x_{k, i}$
      est exogène alors c'est aussi une variable instrumentale d'elle même. En ce sens 
      que non seulement elle vérifie \eqref{vi1} mais elle vérifie forcément \eqref{vi2} 
      (car ayant une corrélation de 1 avec elle même)
    \end{enumerate}
    \end{remark_fr}

    \framebreak

    \item On construit le vecteur des variables instrumentales $\boldz_i$ avec:
    
    \[
       \tilde{\boldz}_i^e = 
       \begin{bmatrix} 
        \tilde{z}_{M+1, i}\\
        \tilde{z}_{M+2, i}\\
        \vdots\\
        \tilde{z}_{K, i}
       \end{bmatrix}
       \ \text{et} \
       \boldz_i = 
       \begin{bmatrix}
        \begin{bmatrix}
            1\\
            \tilde{\boldx}_i^x
        \end{bmatrix}\\ 
        \tilde{\boldz}_i^e
       \end{bmatrix}
       = 
       \begin{bmatrix}
        \boldx_i^x\\
        \tilde{\boldz}_i^e
       \end{bmatrix}
       \begin{array}{ll}
        \left\{ \text{variables exogènes de $\boldx_i$}\right.
         &: \Exp[u_i|x_{k, i}^x] = 0 (k=1,\ldots, M)\\
        \left\{ \text{variables instrumentales}\right.&:\Exp[u_i|z_{k, i}] = 0(k=M +1,\ldots,K)
       \end{array}
    \]
    \item Ce vecteur contient en fait toutes les variables exogènes du modèle. Ce sont 
    ces variables qui assurent l’identification des paramètres du modèle.
    \item $\boldz_i$  est parfois nommé ensemble d’information du modèle.


\framebreak 

\begin{definition_fr}[Modèle linéaire à variables instrumentales]
    Le modèle défini par :
    \begin{align*}
        y_i&=\boldx_i^\prime\bolda_0 + u_i, \ \text{avec} \ \Exp[u_i|\boldz_i]= \Exp[u_i]:=0,
    \end{align*}
    est un modèle linéaire à variables instrumentales.
    La condition d’identification de $\bolda_0$ dans ce modèle est donnée par:
    \begin{align*}
    \Rang\left(\Exp[\boldz\boldx^\prime]\right)= K = \dim(\boldx).
\end{align*}
\end{definition_fr}

\begin{remark_fr}
    La condition d’exogénéité de $\boldz_i$ est définie par $\Exp[u_i|\boldz_i]= 0$, et non par 
    $\Cov[\boldz_i ;u_i ]= \boldzero$.
    Ce n’est pas nécessaire pour un modèle linéaire où $\Cov[\boldz_i ;u_i ]= \boldzero$ suffit
    mais c’est standard et cela simplifie la présentation des hypothèses d’homoscédasticité.
\end{remark_fr}

\framebreak

\item Comme dans le cas où on a construit l’estimateur des MCO de a
on part de la condition d’exogénéité des $\boldz_i$ (et non des $\boldx_i$ comme dans le cas
des MCO), i.e. la condition d’orthogonalité donnée par :
\begin{align*}
    \Exp[u_i\boldz_i]=0 &\Rightarrow \Exp[\boldz_iu_i]=0 \Leftrightarrow 
    \Exp[z_i(y_i-\boldx_i^\prime\bolda_0)]=\boldzero.
\end{align*}
\item On a ici la condition de moment estimante pour $\bolda_0$ est $\Exp[\boldz_i(y_i -\boldx_i\bolda_0)]= \boldzero$. 
Et on a alors:
\begin{align*}
     \Exp[\boldz_i(y_i-\boldx_i^\prime \bolda_0)] = \boldzero \Leftrightarrow \bolda = \bolda_0.
\end{align*}
\item On suppose ici que $\bolda_0$ est l’unique solution en $\bolda$ de $\Exp[\boldz_i(y_i-\boldx^\prime\bolda)]= 0$. 
\item Le principe d’analogie définit l’estimateur de la MM de $\bolda_0$ par :
\begin{align*}
    N^{-1}\sumiN \boldz_i(y_i-\boldx_i^\prime\bolda)=\boldzero_{K\times 1} &\Leftrightarrow \bolda = \hat{\bolda}_N^{MM}.
\end{align*}

\item  L’équation dont $\hat{\bolda}_N^{MM}$ est définie comme la solution en $\bolda$ est en fait un système
de $K$ équations linéaires à $K$ inconnues (les éléments de  $\hat{\bolda}_N^{MM}$). 
Il a solution sous forme explicite. On a:
\begin{align*}
    N^{-1}\sumiN \boldz_i(y_i-\boldx_i^\prime\hat{\bolda}_N^{MM})&=\boldzero_{K\times 1}.
\end{align*}
\item Il est aisé de définir la forme de $\hat{\bolda}_N^{MM}$,
\begin{align*}
N^{-1}\sumiN \boldz_iy_i - \left[N^{-1}\sumiN \boldz_i\boldx_i^\prime \right]\hat{\bolda}_N^{MM}&=\boldzero_{K\times 1},
\end{align*}
qui donne finalement :

\begin{align*}
    \hat{\bolda}_N^{MM} &= \left[N^{-1}\sumiN \boldz_i\boldx_i^\prime \right]^{-1}N^{-1}\sumiN \boldz_iy_i,
\end{align*}
qui définit ce qu’on appelle l’estimateur des VI.
\end{itemize}
\end{frame}
\begin{frame}[allowframebreaks]{Références}
\bibliographystyle{jpe}
 \bibliography{../Biblio}
  \end{frame}



    \end{document}
    